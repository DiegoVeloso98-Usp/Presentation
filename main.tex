\documentclass[usenames,dvipsnames,aspectratio=169]{beamer}
%__________________________________________________________
%
% T H E M E S   A N D   F O R M A T T I N G
%__________________________________________________________
% 
%- - - - - - - - - - - - - - - - - - - - - - - - - - - - - -
%General theme and fonts
%- - - - - - - - - - - - - - - - - - - - - - - - - - - - - -
% Darmstadt, Dresden, Berlin, *Frankfurt*, Ilmenau, Copenhagen, Bergen, Berkeley, Madrid
\usetheme{Frankfurt}

\usecolortheme{default}
\usefonttheme[onlymath]{serif}
\setbeamercolor{footline}{fg=blue}
\setbeamerfont{footline}{series=\bfseries}

% ...existing code...
\setbeamercolor{footline}{fg=white,bg=blue!70!black} % <-- add bg here
\setbeamerfont{footline}{series=\bfseries,size=\tiny}
\setbeamertemplate{footline}{%
  \leavevmode%
  \hbox{%
    \begin{beamercolorbox}[wd=.7\paperwidth,ht=2.5ex,dp=1ex,left]{footline}%
      \hspace{2ex}\usebeamerfont{footline}\insertshorttitle
    \end{beamercolorbox}%
    \begin{beamercolorbox}[wd=.3\paperwidth,ht=2.5ex,dp=1ex,right]{footline}%
      \usebeamerfont{footline}\insertframenumber/\inserttotalframenumber\hspace{2ex}
    \end{beamercolorbox}%
  }%
  \vskip0pt%
}
% ...existing code...

%- - - - - - - - - - - - - - - - - - - - - - - - - - - - - -
%General configurations
%- - - - - - - - - - - - - - - - - - - - - - - - - - - - - -
\setbeamertemplate{items}[square]        % keep/restore square bullets
% \setbeamertemplate{itemize item}[square] % ensure item template is square
% \setbeamertemplate{itemize subitem}[circle]
% \setbeamercolor{itemize item}{fg=black}       % force bullet color (choose black or structure.fg)
% \setbeamercolor{itemize subitem}{fg=black}

\setbeamertemplate{section in toc}[square]
\setbeamertemplate{navigation symbols}{}
\setbeamersize{text margin left=3mm,text margin right=3mm}

\setbeamertemplate{subsection in head/foot}{} % clears the subsection strip
\setbeamertemplate{miniframes}{}    
%- - - - - - - - - - - - - - - - - - - - - - - - - - - - - -
%Break frame of bibliography into parts
%- - - - - - - - - - - - - - - - - - - - - - - - - - - - - -
\setbeamertemplate{frametitle continuation}[from second]
\setbeamertemplate{bibliography item}{}
%__________________________________________________________
%
% P A C K A G E S
%__________________________________________________________
%
%- - - - - - - - - - - - - - - - - - - - - - - - - - - - - -
%Language and encodings
%- - - - - - - - - - - - - - - - - - - - - - - - - - - - - -
\usepackage[english]{babel}
\usepackage[utf8]{inputenc}
%- - - - - - - - - - - - - - - - - - - - - - - - - - - - - -
%Tables and items
%- - - - - - - - - - - - - - - - - - - - - - - - - - - - - -
\usepackage{enumitem}
\usepackage{multicol}
\usepackage{multirow} 
\usepackage{tcolorbox}
\usepackage{booktabs}
\usepackage{mathtools}
\usepackage[table]{xcolor}   % provides \cellcolor (used in your timeline table)
\usepackage{colortbl}        % optional, improves table coloring
\usepackage{adjustbox}
%- - - - - - - - - - - - - - - - - - - - - - - - - - - - - -
%Text formatting
%- - - - - - - - - - - - - - - - - - - - - - - - - - - - - -
\usepackage{indentfirst}
\usepackage{microtype}
\usepackage{color}
\usepackage{ulem}
\usepackage{cancel}
\usepackage{mdframed}
\usepackage{framed}
%- - - - - - - - - - - - - - - - - - - - - - - - - - - - - -
%Figures
%- - - - - - - - - - - - - - - - - - - - - - - - - - - - - -
\usepackage{graphicx}
\usepackage[footnotesize]{caption}
\usepackage{subcaption}
\usepackage{grffile}
\usepackage{ragged2e}
\usepackage{animate}
\usepackage{media9}

%- - - - - - - - - - - - - - - - - - - - - - - - - - - - - -
%Math
%- - - - - - - - - - - - - - - - - - - - - - - - - - - - - -
\usepackage{array}
\usepackage{blkarray}
\usepackage{amsmath, bm, bbm}
\usepackage{amsfonts, latexsym, amssymb}
\usepackage{isomath}
\usepackage{accents}
\usepackage{mathtools}
% \usepackage[linesnumbered, ruled]{algorithm2e}
\usepackage{algorithmicx}
\usepackage{algpseudocode}   % provides algorithmic, \State, \For, \Require, etc.
%- - - - - - - - - - - - - - - - - - - - - - - - - - - - - -
%Placement of objects
%- - - - - - - - - - - - - - - - - - - - - - - - - - - - - -
\usepackage{float}
\usepackage{placeins}
% ...existing code...
\usepackage[font=footnotesize]{caption} % already present in your preamble

% Tight caption / float spacing and appearance
% \captionsetup{
%   font=footnotesize,       % caption font size
%   labelfont=bf,            % label bold
%   skip=2pt,                % space between caption and float (primary control)
%   aboveskip=1pt,
%   belowskip=1pt,
%   singlelinecheck=false,
%   justification=centering
% }
% \captionsetup[figure]{position=top}    % legend (caption) above figure
% \captionsetup[table]{position=top}     % legend above table
% \captionsetup[subfigure]{skip=1pt,font=footnotesize,labelfont=bf}

% % Reduce vertical gaps around floats globally
% \setlength{\textfloatsep}{6pt}   % between floats and text
% \setlength{\floatsep}{6pt}       % between floats
% \setlength{\intextsep}{6pt}      % between in-text floats and text

% % Ensure caption glue also small
% \setlength{\abovecaptionskip}{2pt}
% \setlength{\belowcaptionskip}{2pt}
% ...existing code...
%- - - - - - - - - - - - - - - - - - - - - - - - - - - - - -
%Bibliography, citations and cross referencing
%- - - - - - - - - - - - - - - - - - - - - - - - - - - - - -
% Add ABNT citation support (provides \citeonline etc.). This is inserted
% non-invasively — it will not override existing commands already defined.
\usepackage[alf,abnt-emphasize=bf, abnt-thesis-year=both,
    abnt-repeated-author-omit=yes, abnt-last-names=abnt,
    abnt-etal-cite,abnt-etal-list=3, abnt-etal-text=default,
    abnt-and-type=e,
    abnt-doi=doi, abnt-url-package=none,
    abnt-verbatim-entry=no]{abntex2cite}

% Provide a minimal, safe \fonte and \nota so captions / figure notes behave like your thesis.
% We use \providecommand so we don't override definitions from other classes/packages.
\providecommand{\fontename}{Source}        % default label used by \fonte
\providecommand{\notaname}{Nota}          % default label used by \nota
\providecommand{\ABNTEXfontereduzida}{\footnotesize} % small font used by \fonte

\providecommand{\fonte}[2][\fontename]{%
  \par\ABNTEXfontereduzida\noindent\textbf{#1:}\enspace #2\par
}

\providecommand{\nota}[2][\notaname]{\fonte[#1]{#2}}


%__________________________________________________________
%
% G E N E R I C   C O M M A N D S
%__________________________________________________________
%
%- - - - - - - - - - - - - - - - - - - - - - - - - - - - - - -
% To disable miniframe dots at certain frames with 
% \miniframesoff and \miniframeson
%- - - - - - - - - - - - - - - - - - - - - - - - - - - - - - -
\makeatletter
\let\beamer@writeslidentry@miniframeson=\beamer@writeslidentry
\def\beamer@writeslidentry@miniframesoff{%
\expandafter\beamer@ifempty\expandafter{\beamer@framestartpage}{}
{%else
 % removed \addtocontents commands
 \clearpage\beamer@notesactions%
}
										}
\newcommand*{\miniframeson}{
	\let\beamer@writeslidentry=\beamer@writeslidentry@miniframeson
	}
\newcommand*{\miniframesoff}{
	\let\beamer@writeslidentry=\beamer@writeslidentry@miniframesoff
	}

\beamer@compresstrue

\makeatother
%- - - - - - - - - - - - - - - - - - - - - - - - - - - - - - -
%Command to conviently include a mp4 video
%- - - - - - - - - - - - - - - - - - - - - - - - - - - - - - -
\newcommand{\includemovie}[3]{%
    \includemedia[%
    width=#1,height=#2,%
    activate=pagevisible, transparent,%
    deactivate=pageclose,%
    addresource=#3,%
    flashvars={%
        source=#3 % same path as in addresource!
        &autoPlay=true % default: false; if =true, automatically starts playback after activation (see option ‘activation)’
        &loop=true % if loop=true, media is played in a loop
        %&controlBarAutoHideTimeout=0 %  time span before auto-hide
    }%
    ]{}{VPlayer.swf}%
}% end of the new command
%- - - - - - - - - - - - - - - - - - - - - - - - - - - - - - -
%Environment for flush left equations 
%- - - - - - - - - - - - - - - - - - - - - - - - - - - - - -
\makeatletter
\newenvironment{flequation}
  {\@fleqntrue\begin{equation}}
  {\end{equation}\@fleqnfalse}
\newenvironment{flequation*}
  {\@fleqntrue\begin{equation*}}
  {\end{equation*}\@fleqnfalse}
\makeatother

%- - - - - - - - - - - - - - - - - - - - - - - - - - - - - - -
%Colors
%- - - - - - - - - - - - - - - - - - - - - - - - - - - - - - -
\newcommand{\blue}[1]{\textcolor{blue}{#1}}
\newcommand{\red}[1]{\textcolor{red}{#1}}
\newcommand{\green}[1]{\textcolor{OliveGreen}{#1}}
\newcommand{\black}[1]{\textcolor{black}{#1}}
\newcommand{\yellow}[1]{\textcolor{yellow}{#1}}
\newcommand{\brown}[1]{\textcolor{brown}{#1}}
\newcommand{\purple}[1]{\textcolor{DarkOrchid}{#1}}
\newcommand{\gray}[1]{\textcolor{Gray}{#1}}
\newcommand{\magenta}[1]{\textcolor{magenta}{#1}}
%- - - - - - - - - - - - - - - - - - - - - - - - - - - - - - -
%Derivatives
%- - - - - - - - - - - - - - - - - - - - - - - - - - - - - - -
\newcommand{\pd}[2]{\dfrac{\partial{#1}}{\partial{#2}}}
\newcommand{\pdd}[2]{\dfrac{\partial^2{#1}}{\partial{#2}^2}}
\newcommand{\dd}[2]{\dfrac{d{#1}}{d{#2}}}
\newcommand{\at}[2][\Bigg]{#1|_{#2}}
%- - - - - - - - - - - - - - - - - - - - - - - - - - - - - - -
%Mathematical styles
%- - - - - - - - - - - - - - - - - - - - - - - - - - - - - - -
\newcommand{\matr}[1]{\undertilde{\bm{#1}}}
\newcommand{\tens}[1]{\mathbbm{#1}}
\newcommand{\sett}[1]{\mathcal{#1}\LARGE}
\newcommand{\curss}[1]{\mathscr{#1}}
\newcommand{\norm}[1]{\left\lVert#1\right\rVert}
%- - - - - - - - - - - - - - - - - - - - - - - - - - - - - - -
%Bold letters
%- - - - - - - - - - - - - - - - - - - - - - - - - - - - - - -
\def\bzero{{\bm 0}}
\def\b1{{\bm 1}}
\def\bd{{\bm d}}
\def\be{{\bm e}}
\def\bi{{\bm i}}
\def\bj{{\bm j}}
\def\bn{{\vec{\bm n}}}
\def\br{{\bm r}}
\def\bb{{\vec{\bm b}}}
\def\bu{{\vec{\bm u}}}
\def\bx{{\vec{\bm x}}}
\def\bp{{\vec{\bm p}}}
\def\bL{{\bm L}}
\def\bR{{\bm R}}
\def\bA{{\bm A}}
\def\bF{{\bm F}}
\def\bQ{{\bm Q}}
\def\bq{{\bm q}}
%- - - - - - - - - - - - - - - - - - - - - - - - - - - - - - -
% Mathematical operators
%- - - - - - - - - - - - - - - - - - - - - - - - - - - - - - -
\DeclareMathOperator{\Tr}{Tr}
%- - - - - - - - - - - - - - - - - - - - - - - - - - - - - - -
% Citation commands
%- - - - - - - - - - - - - - - - - - - - - - - - - - - - - - -
\newcommand{\citeblue}[1]{\blue{\cite{#1}}}

%__________________________________________________________
%
% F O R M U L A T I O N   C O M M A N D S
%__________________________________________________________
%
%- - - - - - - - - - - - - - - - - - - - - - - - - - - - - - -
%Variable abbreviations
%- - - - - - - - - - - - - - - - - - - - - - - - - - - - - - -
\def\sig{{\sigma}}
\def\bsig{{\undertilde{\bm \sigma}}}
\def\eps{{\epsilon}}
\def\beps{\bm{\undertilde{\epsilon}}}
\def\epsp{\undertilde{{\bm \epsilon^{\prime}}}}
\def\epsm{{\epsilon_m}}
\def\epseq{{\epsilon_{eq}}}
\def\lm{\lambda}
\def\sigp{\undertilde{{\bsig^{\prime}}}}
\def\es{{\bm e_s}}
\def\ek{{\bm e_k}}
\def\ep{{\bm e_p}}
\def\eq{{\bm e_q}}
\def\gdd{{ \green{\vec{\nabla} d}}}
\def\dd{\red{d}}
\def\qq{\undertilde{{\bm Q}}}
\def\nn{{\vec{\bm n}}}
\def\ww{{\omega(d)}}
\def\gd{{\vec{\bm \nabla}d}}

\def\zp{(\undertilde{\bm{0}}, \dd, \vec{0})}

% None
%__________________________________________________________
%
% D O C U M E N T   I N F O R M A T I O N S
%__________________________________________________________
%
\title{
    Phase-Field modeling of brittle fracture of orthotropic materials
	}

\author{Diego Dias Veloso \and 
\\Ayrton Ribeiro Ferreira (Prof. Dr.) (Advisor)}

\institute{
	Departamento de Engenharia de Estruturas \\
	Escola de Engenharia de São Carlos \\
	Universidade de São Paulo
	} 

\date{
    \today
}
%__________________________________________________________
%
% B E G I N N I N G   O F   T H E   D O C U M E N T
%__________________________________________________________
%
\begin{document}

\justifying

%- - - - - - - - - - - - - - - - - - - - - - - - - - - - - -
%Headline logos in titleframe
%- - - - - - - - - - - - - - - - - - - - - - - - - - - - - -
\setbeamertemplate{headline}[center]
{\setbeamertemplate{footline}{}
\setbeamertemplate{headline}{	}
%- - - - - - - - - - - - - - - - - - - - - - - - - - - - - -
%Do not count tilteframe in numbering
%- - - - - - - - - - - - - - - - - - - - - - - - - - - - - -
\begin{frame}
	\titlepage
\end{frame}}
\addtocounter{framenumber}{-1}
%- - - - - - - - - - - - - - - - - - - - - - - - - - - - - - -
%No headlines/footlines nor numbering in the outline frame
%- - - - - - - - - - - - - - - - - - - - - - - - - - - - - -
{
\setbeamertemplate{footline}{}
\setbeamertemplate{headline}{}
\begin{frame}[allowframebreaks]{Topics}
    \tableofcontents
\end{frame}

}

\addtocounter{framenumber}{-1}
%- - - - - - - - - - - - - - - - - - - - - - - - - - - - - - -
% Head/footlines and numbering for the other frames
%- - - - - - - - - - - - - - - - - - - - - - - - - - - - - - -
\addtobeamertemplate{navigation symbols}{}{
    \usebeamerfont{footline}
    \usebeamercolor[fg]{footline}
    \hspace{1em}
    \insertframenumber/\inserttotalframenumber
} 
%- - - - - - - - - - - - - - - - - - - - - - - - - - - - - - -
% Including other section files
%- - - - - - - - - - - - - - - - - - - - - - - - - - - - - -

\section{Introduction}

\begin{frame}
  \centering
  \vfill
  \begin{beamercolorbox}[sep=14pt,center,rounded=true,shadow=true,wd=\textwidth]{frametitle}
    \usebeamerfont{frametitle}\LARGE Introduction
  \end{beamercolorbox}
  \vfill
\end{frame}


\subsection{Importance of Fracture Mechanics}

\begin{frame}{}
  \frametitle{Importance of Fracture Mechanics}

  The importance of fracture mechanics can be summarized as follows:

  \begin{itemize}[label=\textbullet]
      \item \textbf{Safety}: Predicting failure modes helps prevent catastrophic failures in structures such as bridges, buildings, and aircraft.
      \item \textbf{Material Design}: Insights from fracture mechanics guide the development of new materials with improved toughness and durability.
  \end{itemize}
\end{frame}

\subsection{Anisotropic Fracture}
\begin{frame}{}
  \frametitle{Anisotropic Materials}

  The importance of anisotropic fracture mechanics include:

  \begin{itemize}[label=\textbullet]
      \item \textbf{Directional Dependence} : Many engineering materials exhibit anisotropic properties.
  \end{itemize}


  \begin{figure}
    \centering
    \caption{Examples of anisotropic materials.}
    \begin{subfigure}[b]{0.3\textwidth}
      \includegraphics[width=\linewidth]{figs/1/B2_Bomber.jpg}
      \caption{Military Aircrafts}
    \end{subfigure}\hfill
    \begin{subfigure}[b]{0.3\textwidth}
      \includegraphics[width=\linewidth]{figs/1/jet_turbo_blade.jpg}
      \caption{Jet Engine Turbo Blade}
    \end{subfigure}\hfill
    \begin{subfigure}[b]{0.2\textwidth}
      \includegraphics[width=\linewidth]{figs/1/wind_turbine_blade.png}
      \caption{Wind Turbine}
    \end{subfigure}
    % \vspace{2pt}
    \fonte{Wikipedia.}
  \end{figure}


\end{frame}

% !TEX root = ../main.tex

\section{Objectives}

\begin{frame}
  \centering
  \vfill
  \begin{beamercolorbox}[sep=14pt,center,rounded=true,shadow=true,wd=\textwidth]{frametitle}
    \usebeamerfont{frametitle}\LARGE Objectives
  \end{beamercolorbox}
  \vfill
\end{frame}

\subsection{Novel Contributions}
\begin{frame}{}
    \frametitle{Expected Contributions}
    The main contributions of this work are:
\begin{itemize}[label=\textbullet]
    \item \textbf{Extend the \citeonline{Ferreira2024}} phase-field model to anisotropic materials.
    \item \textbf{Implementation} of a FEM code on top of \textbf{HPC techniques}.
    \item \textbf{Validation} of the model for \textbf{orthotropic materials}.
    \item \textbf{Couple the elastic and fracture toughness anisotropies} on the fracture process.
\end{itemize}

\end{frame}

\subsection{Schedule}
\begin{frame}{}
    \frametitle{Schedule}
    \vspace{-2em}
    \begin{columns}
        \hspace{-7em}
        \begin{column}{0.9\textwidth}
            \begin{table}
                \caption{Activities to be developed.}
                \label{tab:stages}
                \begin{adjustbox}{max width=\textwidth, max totalheight=0.65\textheight, keepaspectratio}
                \begin{tabular}{|p{0.03\linewidth}|p{0.9\linewidth}|}
                    \hline
                    \textbf{No.} & \textbf{Activity description} \\ \hline
                    \cellcolor{cyan!25} 01 & \cellcolor{cyan!25} complete the mandatory course credits for the master’s program \\ \hline
                    \cellcolor{cyan!25} 02 & \cellcolor{cyan!25} bibliographic review \\ \hline
                    \cellcolor{cyan!25} 03 & \cellcolor{cyan!25} study parallelization libraries: PETSc and ParMETIS \\ \hline
                    \cellcolor{cyan!25} 04 & \cellcolor{cyan!25} implementation of the classical isotropic model of \citeonline{Bourdin2000} \\ \hline
                    \cellcolor{cyan!25} 05 & \cellcolor{cyan!25} writing the qualification document \\ \hline
                    \cellcolor{green!30} 06 &  \cellcolor{green!30} qualification exam \\ \hline
                    \cellcolor{cyan!25} 07 & \cellcolor{cyan!25} implementation of the energy decomposition of \citeonline{Ferreira2024} \\ \hline
                    \cellcolor{cyan!25} 08 & \cellcolor{cyan!25} formulation and implementation of the anisotropic elastic constitutive model \\ \hline
                    \cellcolor{cyan!25} 09 & \cellcolor{cyan!25} formulation and implementation of anisotropic fracture resistance \\ \hline
                    10 & period at the Polytechnic University of Milan, Italy \\ \hline
                    11 & development of numerical examples \\ \hline
                    12 & writing a scientific article for an international journal \\ \hline
                    13 & writing the thesis document \\ \hline
                    14 & defense \\ \hline
                \end{tabular}
                \end{adjustbox}
                \caption*{Authors.}
            \end{table}
        \end{column}
        \hspace{-7em}
        \begin{column}{0.45\textwidth}
            \vspace{-5.6em}
            \begin{table}[!htb]
                \centering
                \caption{Activites Timeline.}
                % \vspace{-1em}
                \label{tab:workplan}
                \begin{adjustbox}{max width=\textwidth, max totalheight=0.65\textheight, keepaspectratio}
                    \resizebox{\textwidth}{!}{%
                    \begin{tabular}{|l|*{19}{c|}}
                        \hline
                        & \multicolumn{4}{c|}{\textbf{2024}} & \multicolumn{12}{c|}{\textbf{2025}} & \multicolumn{3}{c|}{\textbf{2026}} \\
                        \hline
                        & \textbf{08} & \textbf{09} & \textbf{10} & \textbf{11} & \textbf{01} & \textbf{02} & \textbf{03} & \textbf{04} & \textbf{05} & \textbf{06} & \textbf{07} & \textbf{08} & \textbf{09} & \textbf{10} & \textbf{11} & \textbf{12} & \textbf{01} & \textbf{02} & \textbf{03} \\
                        \hline
                        Stage 1 & \cellcolor{cyan!25} & \cellcolor{cyan!25} & \cellcolor{cyan!25} & \cellcolor{cyan!25} & & & & & & & & & & & & & & & \\
                        \hline
                        Stage 2 & \cellcolor{cyan!25} & \cellcolor{cyan!25} & \cellcolor{cyan!25} & & & & & & & & & & & & & & & & \\
                        \hline
                        Stage 3 & &  & \cellcolor{cyan!25} & \cellcolor{cyan!25} & & & & & & & & & & & & & & & \\
                        \hline
                        Stage 4 & & &  & \cellcolor{cyan!25} & \cellcolor{cyan!25} & \cellcolor{cyan!25} & & & & & & & & & & & & & \\
                        \hline
                        Stage 5 & & & &  &  & \cellcolor{cyan!25} & \cellcolor{cyan!25} & & & & & & & & & & & & \\
                        \hline
                        Stage 6 & & & & & & & & \cellcolor{green!30} & & & & & & & & & & & \\
                        \hline
                        Stage 7 & & & & & & & & \cellcolor{cyan!25} & \cellcolor{cyan!25} &  & & & & & & & & & \\
                        \hline
                        Stage 8 & & & & & & & & & & \cellcolor{cyan!25} & \cellcolor{cyan!25} &  &  & & & & & & \\
                        \hline
                        Stage 9 & & & & & & & & & & & \cellcolor{cyan!25} & \cellcolor{cyan!25} & \cellcolor{cyan!25} & \cellcolor{cyan!25} &  & & & & \\
                        \hline
                        Stage 10 & & & & & & & & & & & & & \cellcolor{gray!70} & \cellcolor{gray!70} & \cellcolor{gray!70} & \cellcolor{gray!70} & & & \\
                        \hline
                        Stage 11 & & & & & & & & & & & & & & \cellcolor{gray!70} & \cellcolor{gray!70} &  & & & \\
                        \hline
                        Stage 12 & & & & & & & & & & & & & & & \cellcolor{gray!70} & \cellcolor{gray!70} & & & \\
                        \hline
                        Stage 13 & & & & & & & & & & & & & & & & \cellcolor{gray!70} & \cellcolor{gray!70} & \cellcolor{gray!70} & \cellcolor{gray!70} \\
                        \hline
                        Stage 14 & & & & & & & & & & & & & & & & & & & \cellcolor{gray!70} \\
                        \hline
                    \end{tabular}
                    }
                \end{adjustbox}

                \caption*{Authors.}
            \end{table}
        \end{column}
    \end{columns}

\end{frame}


\section{State of the Art}
\begin{frame}
  \centering
  \vfill
  \begin{beamercolorbox}[sep=14pt,center,rounded=true,shadow=true,wd=\textwidth]{frametitle}
    \usebeamerfont{frametitle}\LARGE State of the Art
  \end{beamercolorbox}
  \vfill
\end{frame}




\subsection{Griffith Theory}
\begin{frame}
    \frametitle{Griffith Theory}
  \begin{columns}[t]
    \begin{column}{0.6\textwidth}
      \begin{equation}
        \varPi = U_{elastic} + U_{plastic} + U_{fracture} - W_{ext} \quad ,
      \end{equation}
      \begin{equation}
        \dot{\varPi} = \dot{U}_{elastic} + \dot{U}_{fracture} - \dot{W}_{ext} \quad .
      \end{equation}
      For a closed system, \(\dot{\varPi} = 0\), then the dissipation \(\Phi\) is given by
      \begin{equation}
        \Phi := G\dot{a} = \dot{W}_{ext} - \dot{U}_{elastic} = - \dfrac{d\varPi}{da} \dot{a} \quad ,
      \end{equation}
      \begin{equation}
          G = - \dfrac{d\varPi}{da}
      \end{equation}
      and Griffith's criterion becomes 
      \begin{equation}
          G \geq G_c \quad ,
      \end{equation}

    \end{column}
    \hspace{1em}

    \begin{column}{0.3\textwidth}
      \begin{figure}
        \centering
        \caption{Body undergoing fracture process.}
        \includegraphics[width=\linewidth]{figs/3/fracture_ex.png}
        \fonte{\citeonline{Ashby2012}.}
      \end{figure}
    \end{column}

  \end{columns}
    
\end{frame}

\begin{frame}
    \frametitle{Griffith Theory}
  \begin{columns}[t]
    \begin{column}{0.6\textwidth}
     
      and Griffith's criterion becomes 
      \begin{equation}
          G \geq G_c \quad ,
      \end{equation}
      where \(G_c\) is a material Parameter
      \begin{equation}
      \begin{cases}
        \dot{a} > 0 \rightarrow G = G_c  \\
        \dot{a} = 0 \rightarrow G < G_c 
      \end{cases}  
      \end{equation}
      The critical length of a crack, \(a_c\), is given by
      \begin{equation}
          a_c = \dfrac{2E G_c}{\pi\sigma^2} \quad ,
      \end{equation}

    \end{column}
    \hspace{1em}

    \begin{column}{0.3\textwidth}
      \begin{figure}
        \centering
        \caption{Body undergoing fracture process.}
        \includegraphics[width=\linewidth]{figs/3/fracture_ex.png}
        \fonte{\citeonline{Ashby2012}.}
      \end{figure}
    \end{column}

  \end{columns}
    
\end{frame}

\subsection{Phase-Field Method}
\begin{frame}
    \frametitle{Phase-Field Method}

    \begin{columns}[t]
      \begin{column}{0.6\textwidth}
        The variational recast of the fracture problem leads to
        \begin{equation}
          \varPi_{int} = \underbrace{\int_{\Omega} \psi(\varepsilon)d\Omega}_{\text{Elastic}} 
          + \underbrace{\int_{\Gamma} G_c dA }_{\text{Dissipation}} \quad .
        \end{equation}
        and its regularization by a phase-field \(d\) leads to
        \begin{equation}
          \varPi_{int} = \int_{\Omega} \psi(\bm{\varepsilon},d)d\Omega + \int_{\Omega} \varphi(d,\nabla d) d\Omega \quad .
        \end{equation}
        where
        \begin{equation}
          \begin{cases}
            \psi(\bm{\varepsilon},d) = \omega(d)  \psi_0(\bm{\varepsilon}) \\
              \varphi(d,\nabla d) = G_c\left[w(d) + c_d\left|\nabla d\right|^2\right]  
          \end{cases}
          \end{equation}

      \end{column}
    % \hspace{1em}

      \begin{column}{0.3\textwidth}
        \begin{figure}[H]
            \centering
            \caption{(a) Sharp crack surface \(\Gamma\) and (b) the regularized crack region \(\Gamma_l\).}
            \centering
            \includegraphics[width=\textwidth]{figs/1/sharp_smooth.png}
            \\\caption*{\citeonline{Miehe2010}}
        \end{figure}
      \end{column}

  \end{columns}

\end{frame}

\begin{frame}
    \frametitle{Phase-Field Method}

    \begin{columns}[t]
      \begin{column}{0.45\textwidth}
        Strain energy density
        
        \begin{equation}
            \psi(\bm{\varepsilon},d) = \omega(d)  \psi_0(\bm{\varepsilon})
        \end{equation}

        Quadratic degradation function  
        
        \begin{equation}
            \omega(d) = (1-d)^2
        \end{equation}

    \end{column}
    \hspace{2em}

    \begin{column}{0.45\textwidth}
      Dissipation density
      \begin{equation}
            \varphi(d,\nabla d) = G_c\left[w(d) + c_d\left|\nabla d\right|^2\right]  
      \end{equation}
      where
      \begin{equation}
        w(d) =
        \begin{cases}
          \dfrac{3}{8l_0}d & \text{AT1} \\[0.5ex]
          \dfrac{1}{2l_0}d^2              & \text{AT2}
        \end{cases}
      \end{equation}
      \begin{equation}
          c_d =
        \begin{cases}
          \dfrac{3}{8}l_0 & \text{AT1}\\[0.5ex]
          \dfrac{1}{2}l_0            & \text{AT2}
        \end{cases}
        .
      \end{equation}
    \end{column}
  \end{columns}

\end{frame}


\begin{frame}
    \frametitle{Assymetric Elastic Strain Energy Degradation}
    Symmetric degradation of the strain energy density 
  \begin{figure}[H]
  \centering
  \caption{SEN Mode II - Specimen with interpenetration due to symmetric elastic strain energy degradation.}
    \begin{subfigure}[b]{0.1\textwidth}
      \centering
      \includegraphics[width=\textwidth]{figs/3/sen2_1_nosplit.png}
    \end{subfigure}
    \begin{subfigure}[b]{0.1\textwidth}
      \centering
      \includegraphics[width=\textwidth]{figs/3/sen2_2_nosplit.png}
    \end{subfigure}
    \begin{subfigure}[b]{0.1\textwidth}
      \centering
      \includegraphics[width=\textwidth]{figs/3/sen2_3_nosplit.png}
    \end{subfigure}
    \begin{subfigure}[b]{0.1\textwidth}
        \centering
        \includegraphics[width=\textwidth]{figs/3/sen2_4_nosplit.png}
    \end{subfigure} 
  \\\fonte{Authors.}
  \label{fig:interpenetration}
\end{figure}
Assymetric degradation of the strain energy density 
\(\rightarrow \psi(\bm{\varepsilon},d) = \omega(d)  \psi^+_0(\bm{\varepsilon}) + \psi^-_0(\bm{\varepsilon})\)

\textbf{Volumetric-Deviatoric Split}
\begin{equation}
    \psi^{\pm}_0(\bm{\varepsilon})
    = H^{\pm}\left(\Tr(\bm{\varepsilon})\right)\,\frac{\kappa}{2}\bigl(\Tr(\bm{\varepsilon})\bigr)^2
    + \mu\,\bm{\varepsilon}_D\!:\!\bm{\varepsilon}_D
  \end{equation}

\end{frame}



\subsection{Gradient Split}
\begin{frame}
    \frametitle{Gradient Energy Split}
    \begin{equation}
      \psi(\bm{\varepsilon},d) = 
      \int_{\Omega} \left[\omega(d) + \left[1 - \omega(d)\right]H^-(\sigma_{0n})\right] \psi_0(\bm{\varepsilon})d\Omega\quad ,
    \end{equation}
    \begin{columns}[t]
      \begin{column}{0.45\textwidth}
        where
        \begin{equation}
          \sigma_{0n} = \bm{\sigma} : \bm{n} \otimes \bm{n} \quad .
        \end{equation}
        and the projection direction is given by
        \begin{equation}
          \bm{n} := 
          \begin{cases}
              -\dfrac{\bm{\nabla d}}{\|\bm{\nabla d}\|} & \text{if } \bm{\nabla d} \neq 0 \\[0.5ex]
              0 & \text{if } \bm{\nabla d} = 0 
          \end{cases}
        \end{equation}  
        \begin{equation}
          \sigma_{0n} := 
          \begin{cases}
              \bm{\sigma} : \bm{n} \otimes \bm{n} & \text{if } \bm{\nabla d} \neq 0 \\[0.5ex]
              \Tr(\bm{\sigma}) & \text{if } \bm{\nabla d} = 0 
          \end{cases}
        \end{equation}  
      \end{column}
      \begin{column}{0.4\textwidth}
        \begin{figure}
            \centering
            \caption{Normal to crack direction.}
            \centering
            \includegraphics[width=\textwidth]{figs/3/gradient_split.png}
            \\\caption*{\citeonline{Ferreira2024}}
        \end{figure}
      \end{column}

  \end{columns}

\end{frame}





\section{Computational Implementation}

\begin{frame}
  \centering
  \vfill
  \begin{beamercolorbox}[sep=14pt,center,rounded=true,shadow=true,wd=\textwidth]{frametitle}
    \usebeamerfont{frametitle}\LARGE Computational Implementation
  \end{beamercolorbox}
  \vfill
\end{frame}


\subsection{High-Performance Computing Tools}

\begin{frame}{}
    \frametitle{High-Performance Computing Tools}
    The following high-performance computing tools are utilized in the implementation:
    \begin{itemize}[label=\textbullet]
        \item \textbf{PETSc}: A suite of data structures and routines for the scalable (parallel) solution of scientific applications modeled by partial differential equations.
        \item \textbf{METIS}: A software package for partitioning unstructured graphs and computing fill-reducing orderings of sparse matrices.
        \item \textbf{GMSH}: A 3D finite element mesh generator with built-in pre- and post-processing facilities.
        \item \textbf{HDF5}: A data model, library, and file format for storing, managing and built for fast I/O processing and storage.
        \item \textbf{Paraview}: An open-source, multi-platform data analysis and visualization application.
    \end{itemize}
\end{frame}


% \subsection{PETSc}
\begin{frame}{}
    \frametitle{PETSc}
    PETSc (Portable, Extensible Toolkit for Scientific Computation) is designed for the \textbf{scalable} 
    solution of scientific applications modeled by partial differential equations (\textbf{PDEs}). 
    \begin{figure}[H]
      \centering
        \includegraphics[width=.5\textwidth]{figs/4/petsc.png}
    \end{figure}
    

    \textbf{It is a powerful and convenient interface for}:

    \begin{itemize}[label=\textbullet]
        \item Management of \textbf{parallelism and communication}.
        \item \textbf{Algebraic operations} on distributed vectors (\texttt{Vec}) and matrices(\texttt{Mat}).
        \item Linear (\texttt{KSP}) and nonlinear (\texttt{SNES}) \textbf{solvers} (algorithms (NR, LS, TR, etc) based on \texttt{KSP} objects).
    \end{itemize}


\end{frame}

\begin{frame}{}
    \frametitle{METIS}
    METIS is a library for partitioning and producing fill reducing orderings for finite element meshes.
    \begin{columns}[t]
      \begin{column}{0.45\textwidth}
        \begin{itemize}[label=\textbullet]
          \item \textbf{Fill-reducing}: \texttt{METIS\_NodeND}.
        \end{itemize}
        \begin{figure}[H]
        \centering
        \caption{Sparse matrix representation.}
            \includegraphics[width=1.1\textwidth]{figs/4/sen2_matrix_REFxRCMxMETIS.png}
        \\\caption*{Authors.}
        \label{fig:matrix-sen2}
        \end{figure}
    \end{column}
    % \hspace{2em}
    \begin{column}{0.55\textwidth}
        \begin{itemize}[label=\textbullet]
          \item \textbf{Mesh Partitioning}: \texttt{METIS\_PartMeshDual}.
        \end{itemize}
        \begin{figure}[H]
        \centering
        \caption{Mesh partitioning.}
          \begin{subfigure}[b]{0.4\textwidth}
          \centering
          \includegraphics[width=\textwidth]{figs/4/processScale.png}
        \end{subfigure}
        \\
          \begin{subfigure}[b]{0.15\textwidth}
          \centering
          \includegraphics[width=\textwidth]{figs/4/sen1_mesh.png}
        \end{subfigure}
        \begin{subfigure}[b]{0.15\textwidth}
          \centering
          \includegraphics[width=\textwidth]{figs/4/sen2_mesh.png}
        \end{subfigure}
        \begin{subfigure}[b]{0.15\textwidth}
          \centering
          \includegraphics[width=\textwidth]{figs/4/Lshaped_mesh.png}
        \end{subfigure}
        \begin{subfigure}[b]{0.15\textwidth}
          \centering
          \includegraphics[width=\textwidth]{figs/4/holed_mesh.png}
        \end{subfigure}
        \\
        \begin{subfigure}[b]{0.19\textwidth}
          \centering
          \includegraphics[width=\textwidth]{figs/4/sen1_MetisDualGraph.png}
        \end{subfigure}
        \begin{subfigure}[b]{0.19\textwidth}
          \centering
          \includegraphics[width=\textwidth]{figs/4/sen2_MetisDualGraph.png}
        \end{subfigure}
        \begin{subfigure}[b]{0.15\textwidth}
          \centering
          \includegraphics[width=\textwidth]{figs/4/Lshaped_MetisDualGraph.png}
        \end{subfigure}
        \begin{subfigure}[b]{0.15\textwidth}
          \centering
          \includegraphics[width=\textwidth]{figs/4/holed_MetisDualGraph.png}
        \end{subfigure}
        \\\caption*{Authors.}
        \label{fig:process-partitioning}
      \end{figure}

    \end{column}
  \end{columns}
\end{frame}

% \subsection{Additional Tools}
\begin{frame}{}
    \frametitle{C++, HDF5, Paraview, GMSH, Python}

    \begin{itemize}[label=\textbullet]
      \item \textbf{C++}: The programming language used for the implementation.
      \item \textbf{HDF5}: Built for fast I/O processing and storage.
      \item \textbf{Paraview}: Open-source multiple-platform application for interactive, scientific visualization.
      \item \textbf{GMSH}: A 3D finite element mesh generator with built-in pre- and post-processing facilities.
      \item \textbf{Python}: A programming language used for pre- and post-processing and for the GMSH API.
    \end{itemize}

    \begin{figure}[H]
      \centering
      \begin{subfigure}[b]{0.08\textwidth}
        \centering
        \includegraphics[width=\textwidth]{figs/4/cpp.png}
      \end{subfigure}
      \hspace{1em}
      \begin{subfigure}[b]{0.11\textwidth}
        \centering
        \includegraphics[width=\textwidth]{figs/4/hdf5.png}
      \end{subfigure}
      \hspace{1em}
      \begin{subfigure}[b]{0.37\textwidth}
        \centering
        \includegraphics[width=\textwidth]{figs/4/paraview.png}
      \end{subfigure}
      \hspace{1em}
      \begin{subfigure}[b]{0.09\textwidth}
        \centering
        \includegraphics[width=\textwidth]{figs/4/gmsh_title.png}
      \end{subfigure}
      \hspace{1em}
      \begin{subfigure}[b]{0.07\textwidth}
        \centering
        \includegraphics[width=\textwidth]{figs/4/py.png}
      \end{subfigure}
  \end{figure}

\end{frame}




\subsection{Global Minimization}
\begin{frame}{}
    \frametitle{Solution of the coupled problem}
  \begin{itemize}[label=\textbullet]
    \item \textbf{Monolithic approach} \cite{bharali2022}
  \end{itemize}
  \begin{equation}
      \begin{bmatrix}
      \bm{K}^{uu} & \bm{K}^{ud} \\
      \bm{K}^{du} & \bm{K}^{dd}
      \end{bmatrix}
      \begin{bmatrix}
      \bm{\Delta u} \\
      \Delta d
      \end{bmatrix}
      =
      \begin{bmatrix}
      \bm{r^u} \\
      \bm{r^d}
      \end{bmatrix}
  \end{equation}
  \begin{itemize}[label=\textbullet]
    \item \textbf{Staggered approach}
  \end{itemize}
  \begin{equation}
      \begin{bmatrix}
      \bm{K}^{uu} & 0 \\
      0 & \bm{K}^{dd}
      \end{bmatrix}
      \begin{bmatrix}
      \bm{\Delta u} \\
      \Delta d
      \end{bmatrix}
      =
      \begin{bmatrix}
      \bm{r^u} \\
      \bm{r^d}
      \end{bmatrix}
  \end{equation}
  or, in more detail,
  \begin{equation}
    \begin{cases}
    \partial_u \varPi_{n+1}(\bm{u}^{i+1}_{n+1},d^{i}_{n+1}) = 0 \\
    \partial_d \varPi_{n+1}(\bm{u}^{i+1}_{n+1},d^{i+1}_{n+1}) \left[\Delta d^{i+1}_{n+1}\right] = 0
    , \quad
    \partial_d \varPi_{n+1}(\bm{u}^{i+1}_{n+1},d^{i+1}_{n+1}) \geq 0
    , \quad
    \Delta d^{i+1}_{n+1} \geq 0
    \end{cases}
    .
    \label{eq:alternated_minimization}
\end{equation} 


\end{frame}



\begin{frame}{}
    \frametitle{Staggered approach} 
\begin{algorithmic}  [1] %adds line numbers
\Require Load solution $(\mathbf{u}_n, d_n)$ from step $n$ and boundary conditions $g_{n+1}, t_{n}$ at current step $n$
\State initialize $i \gets 0$
\State set $(\mathbf{u}^{0}, d^{0}) \gets (\mathbf{u}_{n}, d_{n})$
\While{$Re_{\text{stag}} \geq TOL_{\text{stag}}$}
  \State $i \gets i + 1$
  \State given $\,d^{\,i-1} = d_{n} + \Delta d^{\,i-1}\,$, find \,\(\mathbf{u}^{\,i}\) solving \,
  \(
    \partial_{\mathbf{u}}\varPi_{\,n+1}(\mathbf{u}^{\,i}, d^{\,i-1}) = 0
  \)
  \State given \,\(\mathbf{u}^{\,i}\), find \,\(\Delta d^{\,i}\) solving \,
  \(\partial_{d}\varPi_{\,n+1}(u^{\,i}, d_{\,n})[\Delta d^{\,i}]=0\)
  with
  \(\qquad \qquad \partial_{d}\varPi_{\,n+1}(\mathbf{u}^{\,i}, d_{\,n}) \ge 0, \quad \Delta d^{\,i} \ge 0\)
  \State compute \(\left|Re_{\text{stag}} = \partial_{\mathbf{u}}\varPi_{\,n+1}(\mathbf{u}^{\,i}, d^{\,i})[\Delta \mathbf{u}^{\,i}]\right|\)
\EndWhile
\State $(\mathbf{u}_{n+1}, d_{n+1}) \gets (\,\mathbf{u}^{\,i}, d^{\,i}\,)$
\end{algorithmic}

\end{frame}

\subsection{Linear Momentum Solution}

\begin{frame}
\frametitle{Linear Momentum Solution}
% \vspace{-2em}
From Principle of Virtual Work, the weak form is given as:
  \begin{equation}
      \int_{\Omega} \left[\bm{\nabla^s w}:\mathbb{D} \bm{\nabla^s u}\right]dV 
      = \int_{\Gamma_t} \bm{t}\cdot\bm{w}dS 
      + \int_{\mathcal{B}} \bm{b}\cdot\bm{w}dV
      \qquad \forall \text{ \(\bm{w} = 0\) on \(\Gamma_u\)}
  \end{equation}
  \vspace{-3em}
   \begin{columns}[t]
      \begin{column}{0.55\textwidth}
        \begin{equation}
          \begin{cases}
            \bm{u}(x,y) = \bm{N^e}(x,y)\bm{u}^e \\
            \bm{w}(x,y) = \bm{N^e}(x,y)\bm{w}^e 
          \end{cases}
            \forall (x,y) \in \Omega_e
        \end{equation}
        \begin{equation}
          \bm{\varepsilon} = \bm{\nabla^s u} 
          = \bm{B^e}\bm{u^e}
        \end{equation}
        and the system of equations can be written as
        \begin{equation}
            \bm{K}\bm{u} = \bm{F}
        \end{equation}
    \end{column}
    % \hspace{2em}
    \begin{column}{0.46\textwidth}
      \begin{figure}[H]
        \centering
        \caption{Discrete domain.}
        \includegraphics[width=0.4\textwidth]{figs/4/finite_body.png}
        \\\caption*{\citeonline{anand_continuum_2020}.}
      \end{figure}
    \end{column}
  \end{columns}
  \vspace{-1em}
  where
  \vspace{-1em}
  \begin{equation}
    \bm{K} = \sum_{e=1}^n  \left[\int_{\Omega_e} \bm{B}^T_e\mathbb{C}\bm{B}_e d\Omega\right] \bm{u}^e
    \quad \text{and} \quad
    \bm{F} = \sum_{e=1}^n \int_{\varGamma_e} \bm{N}^T_e\bm{t}d\varGamma
            + \sum_{e=1}^n \int_{\Omega_e} \bm{N}^T_e\bm{b}d\Omega
  \end{equation}

\end{frame}

\begin{frame}{}
    \frametitle{Newton Raphson}
    Taylor expansion gives
\begin{equation}
    \bm{R}(\bm{u}^{k+1}) 
    = \bm{R}(\bm{u}^k) 
    + \left[\frac{\partial \bm{R}}{\partial \bm{u}}\right]_{\bm{u}^k}\left(\bm{u}^{k+1} - \bm{u}^k\right)
    + \mathcal{O}\left(\left(\bm{u}^{k+1} - \bm{u}^k\right)^2\right)
\end{equation}
that, after linearization and assuming that \(\bm{R}(\bm{u}^{k+1}) = 0\)
\begin{equation}
    \bm{R}(\bm{u}^k) 
    + \left[\frac{\partial \bm{R}}{\partial \bm{u}}\right]_{\bm{u}^k}\left(\bm{u}^{k+1} - \bm{u}^k\right)
    = 0
    \quad ,
\end{equation}
Rearranging the equation, we can write
\begin{equation}
    \left[\frac{\partial^2 \varPi}{\partial \bm{u}^2}\right]_{\bm{u}^k}\bm{\Delta u}^{k+1} 
    = -\bm{R}(\bm{u}^k)
    \quad \Rightarrow \quad
        \bm{K(\bm{u}^k)}\bm{\Delta u}^{k+1} = -\bm{R}(\bm{u}^k)
\end{equation}
\texttt{PCFactorSetReuseOrdering(PETSC\_TRUE)} and \texttt{KSPSetReusePreconditioner(PETSC\_FALSE)}

\end{frame}

\begin{frame}{}
    \frametitle{Line Search}
    Optimal step size
\begin{equation}
    \bm{u}^{k+1}_{n} = \bm{u}^k + \eta_{n} \bm{\Delta u}^{k+1} \quad .
    \label{eq:line_search}
\end{equation}
assuming that \(\bm{R}(\eta_{n}) \perp \bm{\Delta u}\)
\begin{equation}
    R(\eta_{n}) = \bm{R}(\eta_{n})\cdot\bm{\Delta u} = \bm{R}(\bm{u}^k + \eta_{n}\bm{\Delta u})\cdot\bm{\Delta u}
\end{equation}
that, after some algebra, leads to the following iterative formula for \(\eta\):
\begin{equation}
    \eta_{n+1} = 
    \begin{cases}
        \dfrac{\alpha_{n}}{2} + \sqrt{\left(\dfrac{\alpha_{n}}{2}\right)^2 - \alpha_n } & \text{if } \alpha_n < 0 \\[0.5ex]
        \dfrac{\alpha_{n}}{2} & \text{if } \alpha_n > 0
    \end{cases}
    \quad \rightarrow \quad \text{where} \quad
        \alpha_n = \dfrac{R(\bm{u}^k)}{ R(\eta_n) } = \dfrac{R_0}{R(\eta_n)}
\end{equation}
with a stopping criterion defined by \citeonline{bonet_nonlinear_2016} as 
\begin{equation}
    \left|R(\eta_{n})\right| < \rho \left| R(0)\right| \quad ,
    \qquad \text{which} \quad
    \rho = 0.5
\end{equation}

\end{frame}



\subsection{Damage Solution}
\begin{frame}{}
    \frametitle{Symmetric Linear Complementarity Problem (SLCP)}
    \vspace{-2em}
    \citeonline{Marengo2021} performs a Taylor expansion around \(\hat{\bm{d}}_n\)
    \begin{equation}
    \varPi_{n+1}(\hat{\bm{u}}^i,\hat{\bm{d}})
    = \underbrace{\frac12\,\Delta\hat{\bm{d}}^{\mathsf T}\,Q^i\,\Delta\hat{\bm{d}}
    + \Delta\hat{\bm{d}}^{\mathsf T}\,q^i}_{\text{Term to be minimized}}
    + \varPi_{n+1}(\hat{\bm{u}}^i,\hat{\bm{d}}_n),
\end{equation}
where
\begin{equation}
    \Delta\hat{\bm{d}} = \hat{\bm{d}} - \hat{\bm{d}}_n
    \, ,\quad
    Q^i = \nabla^2_{\hat{\bm{d}}\hat{\bm{d}}} \varPi_{n+1}(\hat{\bm{u}}^i,\hat{\bm{d}}_n)
    \, ,\quad
    q^i = \nabla_{\hat{\bm{d}}}\,\varPi_{n+1}(\hat{\bm{u}}^i,\hat{\bm{d}}_n)\,.
    \label{eq:Qq}
\end{equation}
and finally, the Jacobian (Hessian) \(Q^i\) and Gradient(Residue) \(q^i\) are given by
\begin{equation}
  Q^i := \Psi_e(\hat{\bm{u}}^i) \;+\; G_c\,\Phi_e,
  \qquad
  q^i := Q^i\,\hat{\bm{d}}_n \;-\; \psi_e(\hat{\bm{u}}^i).
\end{equation}


\end{frame}


\begin{frame}{}
    \frametitle{Symmetric Linear Complementarity Problem (SLCP)}
    Where, the constant element dissipation matrix \(\Phi_e\) is given by
\begin{equation}
  \Phi_e := 
  \begin{cases}
    \displaystyle\int_{\Omega_e}
    \left(\frac{3l_0}{8}\,B_{d,e}^{\mathsf T}B_{d,e}\right)
    \,\mathrm{d}\Omega_e & \text{if AT1} \\[2ex]
    \displaystyle\int_{\Omega_e}
    \left(l_0^{-1}\,N_{d,e}^{\mathsf T}N_{d,e}
            + l_0\,B_{d,e}^{\mathsf T}B_{d,e}\right)
    \,\mathrm{d}\Omega_e & \text{if AT2}
  \end{cases}
\end{equation}
and the element free energy matrix \(\Psi_e\) and vector \(\psi_e\) are given by
\begin{equation}
\Psi_e(\hat{\bm{u}}^i_e)
    := \int_{\Omega_e}
    2\,\psi_0^+\bigl(\hat{\bm{u}}^i_e\bigr)\,
    N_{d,e}^{\mathsf T}N_{d,e}
    \,\mathrm d\Omega_e
\end{equation}

\begin{equation}
\psi_e(\hat{\bm{u}}^i_e)
    := 
    \begin{cases}
        \displaystyle \int_{\Omega_e}
        \left[2\,\psi_0^+\bigl(\hat{\bm{u}}^i_e\bigr) - \dfrac{3l_0}{8}\right]\,
        N_{d,e}^{\mathsf T}
        \,\mathrm d\Omega_e. & \text{if AT1} \\[2ex]
        \displaystyle \int_{\Omega_e}
        2\,\psi_0^+\bigl(\hat{\bm{u}}^i_e\bigr)\,
        N_{d,e}^{\mathsf T}
        \,\mathrm d\Omega_e. & \text{if AT2}
    \end{cases}
\end{equation}


\end{frame}

\begin{frame}{}
    \frametitle{Projected Successive Over-Relaxation (PSOR)}
The SLCP is gives rise to the following conditions:
    \begin{equation}
    \partial_d \varPi_{n+1}(\bm{u}_{n+1},d_{n+1}) \left[\Delta d_{n+1}\right]
    , \quad
    \partial_d \varPi_{n+1}(\bm{u}_{n+1},d_{n+1}) \geq 0
    , \quad
    \Delta d_{n+1} \geq 0
    \end{equation}
    which can be solved by a Projected Successive Over-Relaxation (PSOR) iterative scheme
    \begin{equation}
    x^{k+1} = \left(\mathbf{L} + \mathbf{D}\right)^{-1}\left(b - \mathbf{U}x^k\right)
\end{equation}
which, after some algebra, \citeonline{Marengo2021} shows that can be written as
\begin{equation}
    \Delta d^{k}_{r} =
    \left\langle \Delta d^{k-1}_{r} - D^{-1}_{rr}\left[Q_{rc}\Delta d^{k-1}_{c}
    + L_{rc}\left(\Delta d^{k}_{c} - \Delta d^{k-1}_{c}\right)\right] \right\rangle_+
\end{equation}
where
\begin{columns}
  \begin{column}{0.4\textwidth}
    \begin{equation}
      \begin{cases}
        L_{rc} := Q_{r>c} \\[2ex]
        D_{rr} := Q_{r=r}
      \end{cases}
    \end{equation}
  \end{column}
  \begin{column}{0.4\textwidth}
    \begin{equation}
        \langle x \rangle_+ := 
      \begin{cases}
        x & \text{if } x \geq 0 \\[2ex]
        0 & \text{if } x < 0
      \end{cases}
    \end{equation}
  \end{column}
\end{columns} 



\end{frame}



\section{Results}

\begin{frame}
  \centering
  \vfill
  \begin{beamercolorbox}[sep=14pt,center,rounded=true,shadow=true,wd=\textwidth]{frametitle}
    \usebeamerfont{frametitle}\LARGE Results
  \end{beamercolorbox}
  \vfill
\end{frame}


\begin{frame}{}
    \frametitle{Physical Properties for Benchmark Examples}
    \vspace{-1em}
        \begin{figure}
        \centering
        \caption{Benchmarks.}
          \begin{subfigure}[b]{0.19\textwidth}
          \centering
          \includegraphics[width=\textwidth]{figs/5/sen1_mesh.png}
        \end{subfigure}
        \hspace{1.5em}
        \begin{subfigure}[b]{0.19\textwidth}
          \centering
          \includegraphics[width=\textwidth]{figs/5/sen2_mesh.png}
        \end{subfigure}
        \hspace{1.5em}
        \begin{subfigure}[b]{0.19\textwidth}
          \centering
          \includegraphics[width=\textwidth]{figs/5/Lshaped_mesh.png}
        \end{subfigure}
        \hspace{1.5em}
        \begin{subfigure}[b]{0.19\textwidth}
          \centering
          \includegraphics[width=\textwidth]{figs/5/holed_mesh.png}
        \end{subfigure}\\
        \begin{subfigure}[b]{0.19\textwidth}
          \centering
          \includegraphics[width=\textwidth]{figs/5/sen1_bc.png}
        \end{subfigure}
        \hspace{1.5em}
        \begin{subfigure}[b]{0.19\textwidth}
          \centering
          \includegraphics[width=\textwidth]{figs/5/sen2_bc.png}
        \end{subfigure}
        \hspace{1.5em}
        \begin{subfigure}[b]{0.19\textwidth}
          \centering
          \includegraphics[width=\textwidth]{figs/5/Lshaped_bc.png}
        \end{subfigure}
        \hspace{1.5em}
        \begin{subfigure}[b]{0.19\textwidth}
          \centering
          \includegraphics[width=\textwidth]{figs/5/holed_bc.png}
        \end{subfigure}
        \\\caption*{Authors.}
      \end{figure}

\end{frame}

\subsection{Single Edge Notch Mode I}
\begin{frame}{}
    \frametitle{Single Edge Notch Mode I}
    \vspace{-1em}
    \begin{columns}
        \begin{column}{0.5\textwidth}
            \begin{figure}
                \centering
                \caption{Geometry and boundary conditions.}
                \begin{subfigure}[b]{0.45\textwidth}
                    \centering
                    \includegraphics[width=\textwidth]{figs/5/sen1_mesh.png}
                    \end{subfigure}
                \begin{subfigure}[b]{0.45\textwidth}
                    \centering
                    \includegraphics[width=\textwidth]{figs/5/sen1_bc.png}
                \end{subfigure}
                \\\fonte{ \citeonline{Ferreira2024}.}
            \end{figure}
        \end{column}
        \begin{column}{0.55\textwidth}
        Imposed cyclic displacement
            \[
                \Delta \bar{u}_n =
                \begin{cases}
                4 \times 10^{-3} \text{ mm} & \text{if } 1 \leq n \leq 20 \\
                -4 \times 10^{-4} \text{ mm} & \text{if } 21 \leq n \leq 60 \\
                4 \times 10^{-4} \text{ mm} & \text{if } 61 \leq n \leq 80
                \end{cases} 
                .
            \]
        \end{column}
    \end{columns}
    \begin{table}
        \centering
        \caption{Physical properties for benchmark examples.}
        \begin{tabular}{c@{\hspace{3.5em}}c@{\hspace{3.5em}}c@{\hspace{3.5em}}c@{\hspace{3.5em}}c}
        \hline
        Example & $E$ GPa & $\nu$ & $G_c$ N/mm & $\ell_0$ mm   \\
        \hline
        SEN Mode I & 210 & 0.3 & 2.7 & 0.01 \\
        \hline
        \end{tabular}
        \fonte{Authors.}
        \label{tab:numerical_tests}
    \end{table} 
\end{frame}


\begin{frame}{}
  \vspace{-2em}
    \begin{columns}[t] % Add top alignment
        \begin{column}{0.55\textwidth} % Reduce width
            \begin{figure}
            \centering
            \caption{Specimen with load reversal.}
            
            \begin{subfigure}[b]{0.4\textwidth} 
                \centering
                \includegraphics[width=\textwidth]{figs/5/damageScale.png}
            \end{subfigure}

            \begin{subfigure}[b]{0.55\textwidth} 
                \centering
                \animategraphics[loop,autoplay,width=\textwidth]{10}{figs/5/sen1/result.}{0000}{0079}            
            \end{subfigure}
            \\\fonte{Authors.}
            \end{figure}
        \end{column}
        \hspace{-3em} 
        \begin{column}{0.5\textwidth} 
          \vspace{2em} 
          \begin{figure}
            \caption{Reaction forces.}
            \centering
            \animategraphics[loop,autoplay,width=0.9\textwidth]{10}{figs/5/sen1/frame_}{0002}{0081}            
            \\\fonte{Authors.}
          \end{figure}
        \end{column}
      \end{columns}
\end{frame}


\subsection{Single Edge Notch Mode II}
\begin{frame}{}
    \frametitle{Single Edge Notch Mode II}
    \vspace{-1em}
    \begin{columns}
        \begin{column}{0.5\textwidth}
            \begin{figure}
                \centering
                \caption{Geometry and boundary conditions.}
                \begin{subfigure}[b]{0.45\textwidth}
                    \centering
                    \includegraphics[width=\textwidth]{figs/5/sen2_mesh.png}
                \end{subfigure}
                \begin{subfigure}[b]{0.45\textwidth}
                    \centering
                    \includegraphics[width=\textwidth]{figs/5/sen2_bc.png}
                \end{subfigure}
                \\\fonte{ \citeonline{Ferreira2024}.}
            \end{figure}
        \end{column}
        \begin{column}{0.55\textwidth}
            The imposed cyclic loading for this case is:
            \[
                \Delta \bar{u}_n =
                \begin{cases}
                1 \times 10^{-3} \text{ mm} & \text{if } 1 \leq n \leq 6 \\
                3 \times 10^{-4} \text{ mm} & \text{if } 7 \leq n \leq 26  \\
                -3 \times 10^{-4} \text{ mm} & \text{if } 27 \leq n \leq 106  \\
                3 \times 10^{-4} \text{ mm} & \text{if } 107 \leq n \leq 146 
                \end{cases} 
                 \quad .
            \]
        \end{column}
    \end{columns}
    \begin{table}
        \centering
        \caption{Physical properties for benchmark examples.}
        \begin{tabular}{c@{\hspace{3.5em}}c@{\hspace{3.5em}}c@{\hspace{3.5em}}c@{\hspace{3.5em}}c}
        \hline
        Example & $E$ GPa & $\nu$ & $G_c$ N/mm & $\ell_0$ mm   \\
        \hline
        SEN Mode II & 210 & 0.3 & 2.7 & 0.01 \\
        \hline
        \end{tabular}
        \fonte{Authors.}
        \label{tab:numerical_tests}
    \end{table} 
\end{frame}

\begin{frame}{}
  \vspace{-2em}
    \begin{columns}[t] % Add top alignment
        \begin{column}{0.55\textwidth} % Reduce width
            \begin{figure}
            \centering
            \caption{Specimen with load reversal.}
            
            \begin{subfigure}[b]{0.4\textwidth} 
                \centering
                \includegraphics[width=\textwidth]{figs/5/damageScale.png}
            \end{subfigure}

            \begin{subfigure}[b]{0.55\textwidth} 
                \centering
                \animategraphics[loop,autoplay,width=\textwidth]{10}{figs/5/sen2/result.}{0000}{0145}            
            \end{subfigure}
            \\\fonte{Authors.}
            \end{figure}
        \end{column}
        \hspace{-3em} 
        \begin{column}{0.5\textwidth} 
          \vspace{2em} 
          \begin{figure}
            \caption{Reaction forces.}
            \centering
            \animategraphics[loop,autoplay,width=0.9\textwidth]{10}{figs/5/sen2/frame_}{0002}{0147}            
            \\\fonte{Authors.}
          \end{figure}
        \end{column}
      \end{columns}
\end{frame}


\subsection{L-Shaped Specimen}
\begin{frame}{}
    \frametitle{L-Shaped Specimen}
    \vspace{-1em}
    \begin{columns}
        \begin{column}{0.5\textwidth}
            \begin{figure}
                \centering
                \caption{Geometry and boundary conditions.}
                \begin{subfigure}[b]{0.42\textwidth}
                    \centering
                    \includegraphics[width=\textwidth]{figs/5/Lshaped_mesh.png}
                \end{subfigure}
                \begin{subfigure}[b]{0.42\textwidth}
                    \centering
                    \includegraphics[width=\textwidth]{figs/5/Lshaped_bc.png}
                \end{subfigure}
                \caption{ \citeonline{Ferreira2024}.}
            \end{figure}
        \end{column}
        \begin{column}{0.55\textwidth}
            Subjected to the following displacement history:
            \[
                \Delta \bar{u}_n =
                \begin{cases}
                1 \times 10^{-2} \text{ mm} & \text{if } 1 \leq n \leq 36  \\
                -3 \times 10^{-2} \text{ mm} & \text{if } 37 \leq n \leq 48  \\
                -1 \times 10^{-2} \text{ mm} & \text{if } 49 \leq n \leq 84  \\
                3 \times 10^{-2} \text{ mm} & \text{if } 85 \leq n \leq 96  
                \end{cases} 
                \quad .
            \]
        \end{column}
    \end{columns}   
    \vspace{-1em}
    \begin{table}
        \centering
        \caption{Physical properties for benchmark examples.}
        \begin{tabular}{c@{\hspace{3.5em}}c@{\hspace{3.5em}}c@{\hspace{3.5em}}c@{\hspace{3.5em}}c}
        \hline
        Example & $E$ GPa & $\nu$ & $G_c$ N/mm & $\ell_0$ mm   \\
        \hline
        L-Shaped & 25.85 & 0.18 & 0.095 & 5.0 \\
        \hline
        \end{tabular}
        \fonte{Authors.}
        \label{tab:numerical_tests}
    \end{table} 
\end{frame}

\begin{frame}{}
  \vspace{-2em}
    \begin{columns}[t] % Add top alignment
        \begin{column}{0.55\textwidth} % Reduce width
            \begin{figure}
            \centering
            \caption{Specimen with load reversal.}
            
            \begin{subfigure}[b]{0.4\textwidth} 
                \centering
                \includegraphics[width=\textwidth]{figs/5/damageScale.png}
            \end{subfigure}

            \begin{subfigure}[b]{0.55\textwidth} 
                \centering
                \animategraphics[loop,autoplay,width=\textwidth]{10}{figs/5/Lshaped/result.}{0000}{0095}            
            \end{subfigure}
            \\\fonte{Authors.}
            \end{figure}
        \end{column}
        \hspace{-3em} 
        \begin{column}{0.5\textwidth} 
          \vspace{2em} 
          \begin{figure}
            \caption{Reaction forces.}
            \centering
            \animategraphics[loop,autoplay,width=0.9\textwidth]{10}{figs/5/Lshaped/frame_}{0002}{0097}            
            \\\fonte{Authors.}
          \end{figure}
        \end{column}
      \end{columns}
\end{frame}


\subsection{Holed Plate Specimen}
\begin{frame}{}
    \frametitle{Holed Plate Specimen}
    \vspace{-1em}
    \begin{columns}
        \begin{column}{0.5\textwidth}
            \begin{figure}
                \centering
                \caption{Geometry and boundary conditions.}
                \begin{subfigure}[b]{0.45\textwidth}
                    \centering
                    \includegraphics[width=\textwidth]{figs/5/holed_mesh.png}
                \end{subfigure}
                \begin{subfigure}[b]{0.45\textwidth}
                    \centering
                    \includegraphics[width=\textwidth]{figs/5/holed_bc.png}
                \end{subfigure}
                \\\caption{ \citeonline{Ferreira2024}.}
            \end{figure}
        \end{column}    
        \begin{column}{0.55\textwidth}
            The imposed cyclic loading is specified as:
            \[
                \Delta \bar{u}_n =
                \begin{cases}
                1 \times 10^{-2} \text{ mm} & \text{if } 1 \leq n \leq 301 \\
                3 \times 10^{-2} \text{ mm} & \text{if } 302 \leq n \leq 903 
                \end{cases} 
                \quad .
            \]
        \end{column}
    \end{columns}
    \vspace{-1em}
    \begin{table}
        \centering
        \caption{Physical properties for benchmark examples.}
        \begin{tabular}{c@{\hspace{3.5em}}c@{\hspace{3.5em}}c@{\hspace{3.5em}}c@{\hspace{3.5em}}c}
        \hline
        Example & $E$ GPa & $\nu$ & $G_c$ N/mm & $\ell_0$ mm   \\
        \hline
        Holed Plate & 210 & 0.3 & 2.7 & 0.02 \\
        \hline
        \end{tabular}
        \fonte{Authors.}
        \label{tab:numerical_tests}
    \end{table} 
\end{frame}


\begin{frame}{}
  \vspace{-2em}
    \begin{columns}[t] % Add top alignment
        \begin{column}{0.55\textwidth} % Reduce width
            \begin{figure}
            \centering
            \caption{Specimen with load reversal.}
            
            \begin{subfigure}[b]{0.4\textwidth} 
                \centering
                \includegraphics[width=\textwidth]{figs/5/damageScale.png}
            \end{subfigure}

            \begin{subfigure}[b]{0.55\textwidth} 
                \centering
                \animategraphics[loop,autoplay,width=\textwidth]{10}{figs/5/holed/result.}{0003}{0102}            
            \end{subfigure}
            \\\fonte{Authors.}
            \end{figure}
        \end{column}
        \hspace{-3em} 
        \begin{column}{0.5\textwidth} 
          \vspace{2em} 
          \begin{figure}
            \caption{Reaction forces.}
            \centering
            \animategraphics[loop,autoplay,width=0.9\textwidth]{10}{figs/5/holed/frame_}{0002}{0101}            
            \\\fonte{Authors.}
          \end{figure}
        \end{column}
      \end{columns}
\end{frame}




\section{Partial Conclusions}

\begin{frame}
  \centering
  \vfill
  \begin{beamercolorbox}[sep=14pt,center,rounded=true,shadow=true,wd=\textwidth]{frametitle}
    \usebeamerfont{frametitle}\LARGE Partial Conclusions
  \end{beamercolorbox}
  \vfill
\end{frame}




\begin{frame}{}
    \frametitle{Results}

    The main contributions of this thesis are:
\begin{itemize}[label=\textbullet]
  \item Anisotropic phase-field damage formulation for elastic and fracture \textbf{anisotropy}.
  \item A parallel FEM code using \textbf{HPC} techniques and tools.
  \item Identification of model-implementation \textbf{limitations} and proposals for \textbf{extensions} (e.g., Drucker-Prager criteria and alternative energy splits).
\end{itemize}

The following key observations were made:

\begin{itemize}[label=\textbullet]
  \item The model closely \textbf{approximated the crack path and peak load} within the \textit{Vol-Dev} model reference.
  \item The model \textbf{captured the load reversal} behavior accurately.
  \item The model is directly \textbf{extensible to anisotropic materials}.
\end{itemize}

\end{frame}



\begin{frame}{}
    \frametitle{Results}

Despite the successful implementation and validation, some limitations remain:
\begin{itemize}[label=\textbullet]
  \item \textbf{Fixing \(H^-\) during staggered iterations.} In the current staggered solution strategy the negative Heaviside field \(H^-\) is held fixed until the minimum residual is reached, unlike the volumetric–deviatoric (Vol–Dev) split where \(H^-\) is allowed to evolve.
  \item \textbf{Energy split formulation and compressive fracture.} The present energy split does not represent fracture initiation under primarily compressive stress states. A possible remedy is to incorporate a Drucker–Prager type criterion.
  \item \textbf{Solver and scalability improvements.} While MPI provides good scalability, introducing better preconditioners and iterative solvers could further improve performance.
\end{itemize}

\end{frame}


\section{Next Steps}

\begin{frame}
  \centering
  \vfill
  \begin{beamercolorbox}[sep=14pt,center,rounded=true,shadow=true,wd=\textwidth]{frametitle}
    \usebeamerfont{frametitle}\LARGE Next Steps
  \end{beamercolorbox}
  \vfill
\end{frame}


\begin{frame}{}
    \frametitle{Next Steps}
The next steps to be taken are:
    \begin{itemize}[label=\textbullet]
    \item Validate the implementation of the transversely isotropic and orthotropic anisotropies of the constitutive tensor;
    \item Develop validation tests for transversely isotropic and orthotropic materials symmetries;
    \item Validate the implementation of the weak anisotropy model;
    \item Write an international paper for a high impact journal;
    \item Implement an efficient solver for the implemented strong anisotropy model;
    \item \underline{If possible:} Development of at least one 3D example with trigonal symmetry.
\end{itemize}

\end{frame}


%- - - - - - - - - - - - - - - - - - - - - - - - - - - - - - -
% Preparing reference frames
%- - - - - - - - - - - - - - - - - - - - - - - - - - - - - - -
\section[]{References and Acknowledgments}
\miniframesoff

\begin{frame}[t, allowframebreaks]
	\frametitle{References}
	\bibliography{bibliography}
\end{frame}

\begin{frame}
	\frametitle{Acknowledgments}
  \begingroup
  % use the theme's structure color for a light "beamer blue" background
  \setbeamercolor{background canvas}{bg=structure!10}
  \setbeamercolor{frametitle}{fg=structure!85!black,bg=structure!10}
  
  \begin{center}
    % {\color{structure!85!black}\Huge\textbf{Acknowledgments}}
    
    Prof. Dr. Ayrton Ribeiro Ferreira (Advisor) \\
    \vspace{1em}

    Prof. Dr. Marco Lúcio Bittencourt (External Examiner) \\
    \vspace{1em}
    Prof. Dr. Rodolfo André Kuche Sanches (Internal Examiner) \\
    \vspace{1em}
    Matt Knepley (PETSc Team) \\
    \vspace{1em}
    CNPq - Conselho Nacional de Desenvolvimento Científico e Tecnológico \\
    \vspace{1em}
    FAPESP - Fundação de Amparo à Pesquisa do Estado de São Paulo \\
    \vspace{1em}
  \end{center}
  
  \vfill
  
  \hfill
  \begin{minipage}{0.6\textwidth}
    \raggedleft
    {\color{structure!60!black}\footnotesize\textit{"Extraordinary claims require \\ extraordinary evidence."}}
    
    {\color{structure!85!black}\footnotesize\textit{--- Carl Sagan}}
  \end{minipage}
  \hspace{1cm}
  
  \vspace{0.8cm}
  \endgroup
\end{frame}

\section{Current Work}

\subsection{Weak Anisotropy}
\begin{frame}{}
    \frametitle{Current Work}
    Coupled elastic and weak fracture toughness anisotropies in the phase-field method.
    \begin{figure}
    \centering
    \caption{Geometry and boundary conditions.}
    \begin{subfigure}[b]{0.45\textwidth}
        \centering
        \includegraphics[width=\textwidth]{figs/Current/holed_bar_geometry.png}
    \end{subfigure}
    \begin{subfigure}[b]{0.45\textwidth}
        \centering
        \includegraphics[width=\textwidth]{figs/Current/holed_bar_mesh.png}
    \end{subfigure}
    \\
    \begin{subfigure}[b]{0.45\textwidth}
        \centering
        \includegraphics[width=\textwidth]{figs/Current/holed_ortho_weak_15.png}
        \caption*{15\textdegree{} rotation.}
    \end{subfigure}
    \begin{subfigure}[b]{0.45\textwidth}
        \centering
        \includegraphics[width=\textwidth]{figs/Current/holed_ortho_weak_30.png}
        \caption*{30\textdegree{} rotation.}
    \end{subfigure}
    \\\caption*{Authors.}
\end{figure}
\end{frame}

\begin{frame}{}
    \frametitle{Current Work}
    Coupled elastic and weak fracture toughness anisotropies in the phase-field method.
    \begin{figure}
    \centering
    \caption{Reaction plots for several fiber orientations.}
    \begin{subfigure}[b]{0.4\textwidth}
        \centering
        \includegraphics[width=\textwidth]{figs/Current/holed_combined,30,15,0.pdf}
    \end{subfigure}
    \begin{subfigure}[b]{0.4\textwidth}
        \centering
        \includegraphics[width=\textwidth]{figs/Current/holed_combined,90,60,45.pdf}
    \end{subfigure}
    \\\caption*{Authors.}
\end{figure}
\end{frame}

\begin{frame}{}
    \frametitle{Current Work}
    Coupled elastic and weak fracture toughness anisotropies in the phase-field method.
    \begin{figure}
    \centering
    \caption{Geometry and boundary conditions.}
    \begin{subfigure}[b]{0.3\textwidth}
        \centering
        \includegraphics[width=\textwidth]{figs/Current/sen1_ortho_mesh.png}
        \caption*{Mesh.}
    \end{subfigure}
    \begin{subfigure}[b]{0.3\textwidth}
        \centering
        \includegraphics[width=\textwidth]{figs/Current/sen1_ortho_weak_15.png}
        \caption*{15\textdegree{} rotation.}
    \end{subfigure}
    \begin{subfigure}[b]{0.3\textwidth}
        \centering
        \includegraphics[width=\textwidth]{figs/Current/sen1_ortho_weak_30.png}
        \caption*{30\textdegree{} rotation .}
    \end{subfigure}
    \\\caption*{Authors.}
\end{figure}

\end{frame}

\subsection{Strong Anisotropy}
\begin{frame}{}
    \frametitle{Current Work}
    Strong anisotropy in the phase-field fracture method.
    \begin{figure}
    \caption{Geometry and boundary conditions.}
    \begin{subfigure}[b]{0.25\textwidth}
        \centering
        \includegraphics[width=\textwidth]{figs/Current/sen1_iso_strong15.png}
        \caption*{Elastic Iso + PF Strong Aniso with 15\textdegree{} rotation.}
    \end{subfigure}
    \begin{subfigure}[b]{0.25\textwidth}
        \centering
        \includegraphics[width=\textwidth]{figs/Current/sen1_ortho_Iso4th.png}
        \caption*{Elastic Ortho with 30\textdegree{} rotation + PF 4th order Iso.}
    \end{subfigure}
    \\\caption*{Authors.}
\end{figure}


    
\end{frame}



\end{document}
