
\section{Introduction}

\begin{frame}
  \centering
  \vfill
  \begin{beamercolorbox}[sep=14pt,center,rounded=true,shadow=true,wd=\textwidth]{frametitle}
    \usebeamerfont{frametitle}\LARGE Introduction
  \end{beamercolorbox}
  \vfill
\end{frame}


\subsection{Importance of Fracture Mechanics}

\begin{frame}{}
  \frametitle{Importance of Fracture Mechanics}

  The importance of fracture mechanics can be summarized as follows:

  \begin{itemize}[label=\textbullet]
      \item \textbf{Safety}: Predicting failure modes helps prevent catastrophic failures in structures such as bridges, buildings, and aircraft.
      \item \textbf{Material Design}: Insights from fracture mechanics guide the development of new materials with improved toughness and durability.
  \end{itemize}
\end{frame}

\subsection{Anisotropic Fracture}
\begin{frame}{}
  \frametitle{Anisotropic Materials}

  The importance of anisotropic fracture mechanics include:

  \begin{itemize}[label=\textbullet]
      \item \textbf{Directional Dependence} : Many engineering materials exhibit anisotropic properties.
  \end{itemize}


  \begin{figure}
    \centering
    \caption{Examples of anisotropic materials.}
    \begin{subfigure}[b]{0.3\textwidth}
      \includegraphics[width=\linewidth]{figs/1/B2_Bomber.jpg}
      \caption{Military Aircrafts}
    \end{subfigure}\hfill
    \begin{subfigure}[b]{0.3\textwidth}
      \includegraphics[width=\linewidth]{figs/1/jet_turbo_blade.jpg}
      \caption{Jet Engine Turbo Blade}
    \end{subfigure}\hfill
    \begin{subfigure}[b]{0.2\textwidth}
      \includegraphics[width=\linewidth]{figs/1/wind_turbine_blade.png}
      \caption{Wind Turbine}
    \end{subfigure}
    % \vspace{2pt}
    \fonte{Wikipedia.}
  \end{figure}


\end{frame}
