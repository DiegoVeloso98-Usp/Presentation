
\section{State of the Art}
\begin{frame}
  \centering
  \vfill
  \begin{beamercolorbox}[sep=14pt,center,rounded=true,shadow=true,wd=\textwidth]{frametitle}
    \usebeamerfont{frametitle}\LARGE State of the Art
  \end{beamercolorbox}
  \vfill
\end{frame}




\subsection{Griffith Theory}
\begin{frame}
    \frametitle{Griffith Theory}
      \vspace{-1.5em}
    \begin{equation}
      \varPi = U_{elastic} + {\color{red}U_{plastic}} + U_{fracture} - W_{ext} \quad \underbrace{\Longrightarrow}_{Brittle} \quad \dot{\varPi} = \dot{U}_{elastic} + \dot{U}_{fracture} - \dot{W}_{ext} 
    \end{equation}    
  \begin{columns}[t]
    \begin{column}{0.6\textwidth}
      For a closed system, \(\dot{\varPi} = 0\), then the \textbf{dissipation} \(\Phi\) is 
      \begin{equation*}
        \Phi = \dot{W}_{ext} - \dot{U}_{elastic} := G\dot{a} =  - \dfrac{d\varPi}{da} \dot{a} \quad , \quad G = - \dfrac{d\varPi}{da}
      \end{equation*}
      % \begin{equation}
      %     G = - \dfrac{d\varPi}{da}
      % \end{equation}
      and Griffith's criterion becomes 
      \begin{equation}
        G \geq G_c \quad , \text{and} \quad
        {\color{red}
        \begin{cases}
          \dot{a} > 0 \rightarrow G = G_c  \\
          \dot{a} = 0 \rightarrow G < G_c 
        \end{cases} 
        } 
      \end{equation}
      \textbf{Critical length} of a crack, \(a_c\), is given by \(a_c = \dfrac{2E G_c}{\pi\sigma^2} .\)
    \end{column}
    \hspace{1em}
    \begin{column}{0.35\textwidth}
      \vspace{-3.0em}
      \begin{figure}
        \centering
        \caption{Body undergoing fracture.}
        \vspace{-1.0em}
        \includegraphics[width=\textwidth]{figs/3/fracture_ex.png}
        \citeonline{Ashby2012}.
      \end{figure}
    \end{column}
  \end{columns}
  
\end{frame}

\subsection{Phase-Field Method}
\begin{frame}
    \frametitle{Phase-Field Method}
    \begin{columns}[t]
      \begin{column}{0.6\textwidth}
        The variational recast done by \citeonline{Francfort1998}
        \begin{equation}
          \varPi_{int} = \underbrace{\int_{\Omega} \psi(\varepsilon)d\Omega}_{\text{Elastic}} 
          + \underbrace{\int_{\Gamma} G_c dA }_{\text{Dissipation}} \quad .
        \end{equation}
        and its regularization by a phase-field \(d\) performed by \citeonline{Bourdin2000}
        \begin{equation}
          \varPi_{int} = \int_{\Omega} \psi(\bm{\varepsilon},{\color{red}\bm{d}})d\Omega + \int_{\Omega} \varphi({\color{red}\bm{d},\bm{\nabla d}}) d\Omega \quad .
        \end{equation}
        \begin{equation}
          \begin{cases}
            \psi(\bm{\varepsilon},d) = \omega(d)  \psi_0(\bm{\varepsilon}) \\
              \varphi(d,\nabla d) = G_c\left[w(d) + c_d\left|\nabla d\right|^2\right]  
          \end{cases}
          \end{equation}

      \end{column}
    % \hspace{1em}

      \begin{column}{0.3\textwidth}
        \begin{figure}[H]
            \centering
            \caption{(a) Sharp crack surface \(\Gamma\) and (b) the regularized crack region \(\Gamma_l\).}
            \centering
            \includegraphics[width=\textwidth]{figs/1/sharp_smooth.png}
            \\\caption*{\citeonline{Miehe2010}}
        \end{figure}
      \end{column}

  \end{columns}

\end{frame}

\begin{frame}
    \frametitle{Phase-Field Method}
    % \vspace{-2em}
    \begin{columns}[t]
      \begin{column}{0.5\textwidth}
        The Strain and dissipation energy densities
        \[
          \begin{aligned}
            {\color{red}\psi(\bm{\varepsilon},d)} &{\color{red}= \omega(d)\,\psi_0(\bm{\varepsilon})}, \quad \omega(d)=(1-d)^2\\[0.5ex]
            \varphi(d,\nabla d) &= G_c\bigl[w(d) + c_d\lvert\nabla d\rvert^2\bigr]
          \end{aligned}
        \]

        where
        \begin{equation*}
          w(d) =
          \begin{cases}
            \dfrac{3}{8l_0}d & \text{AT1} \\[0.5ex]
            \dfrac{1}{2l_0}d^2              & \text{AT2}
          \end{cases}
            c_d =
          \begin{cases}
            \dfrac{3}{8}l_0 & \text{AT1}\\[0.5ex]
            \dfrac{1}{2}l_0            & \text{AT2}
          \end{cases}
        \end{equation*}.        
    \end{column}
    \begin{column}{0.5\textwidth}
      \vspace{-2em}
        \begin{figure}[H]
        \centering
        \caption{Interpenetration due to symmetric degradation.}
          \begin{subfigure}[b]{0.25\textwidth}
            \centering
            \includegraphics[width=\textwidth]{figs/3/sen2_1_nosplit.png}
          \end{subfigure}
          \hspace{-1.0em}
          \begin{subfigure}[b]{0.25\textwidth}
            \centering
            \includegraphics[width=\textwidth]{figs/3/sen2_2_nosplit.png}
          \end{subfigure}
          \hspace{-1.0em}
          \begin{subfigure}[b]{0.25\textwidth}
            \centering
            \includegraphics[width=\textwidth]{figs/3/sen2_3_nosplit.png}
          \end{subfigure}
          \hspace{-1.0em}
          \begin{subfigure}[b]{0.25\textwidth}
              \centering
              \includegraphics[width=\textwidth]{figs/3/sen2_4_nosplit.png}
          \end{subfigure} 
        \\\fonte{Authors.}
        \label{fig:interpenetration}
      \end{figure}
      Assymetric degradation of the strain energy
      \begin{equation*}
        {\color{red}
         \psi(\bm{\varepsilon},d) = \omega(d)  \psi^+_0(\bm{\varepsilon}) + \psi^-_0(\bm{\varepsilon}) }
      \end{equation*}
    \end{column}
  \end{columns}
  \textbf{Volumetric-Deviatoric Split}
  \begin{equation}
    \psi^{\pm}_0(\bm{\varepsilon})
    = H^{\pm}\left(\Tr(\bm{\varepsilon})\right)\,\frac{\kappa}{2}\bigl(\Tr(\bm{\varepsilon})\bigr)^2
    + \mu\,\bm{\varepsilon}_D\!:\!\bm{\varepsilon}_D
  \end{equation}

\end{frame}


\subsection{Gradient Split}
\begin{frame}
    \frametitle{Gradient Energy Split}
    \begin{equation}
      \psi(\bm{\varepsilon},d) = 
      \int_{\Omega} \left[\omega(d) + \left[1 - \omega(d)\right]{\color{red}H^-(\sigma_{0n})}\right] \psi_0(\bm{\varepsilon})d\Omega\quad ,
    \end{equation}
    \begin{columns}[t]
      \begin{column}{0.45\textwidth}
        where
        \begin{equation}
          \sigma_{0n} = \bm{\sigma} : \bm{n} \otimes \bm{n} \quad .
        \end{equation}
        and the projection direction is given by
        \begin{equation}
          \bm{n} := 
          \begin{cases}
              -\dfrac{\bm{\nabla d}}{\|\bm{\nabla d}\|} & \text{if } \bm{\nabla d} \neq 0 \\[0.5ex]
              0 & \text{if } \bm{\nabla d} = 0 
          \end{cases}
        \end{equation}  
        \begin{equation}
          \sigma_{0n} := 
          \begin{cases}
              \bm{\sigma} : \bm{n} \otimes \bm{n} & \text{if } \bm{\nabla d} \neq 0 \\[0.5ex]
              \Tr(\bm{\sigma}) & \text{if } \bm{\nabla d} = 0 
          \end{cases}
        \end{equation}  
      \end{column}
      \begin{column}{0.4\textwidth}
        \vspace{-2.5em}
        \begin{figure}
            \centering
            \caption{Normal to crack direction.}
            \centering
            \includegraphics[width=\textwidth]{figs/3/gradient_split.png}
            \\\caption*{\citeonline{Ferreira2024}}
        \end{figure}
      \end{column}

  \end{columns}

\end{frame}




