\section{Results}

\begin{frame}
  \centering
  \vfill
  \begin{beamercolorbox}[sep=14pt,center,rounded=true,shadow=true,wd=\textwidth]{frametitle}
    \usebeamerfont{frametitle}\LARGE Results
  \end{beamercolorbox}
  \vfill
\end{frame}


\begin{frame}{}
    \frametitle{Physical Properties for Benchmark Examples}
    \vspace{-1em}
        \begin{figure}
        \centering
        \caption{Benchmarks.}
          \begin{subfigure}[b]{0.19\textwidth}
          \centering
          \includegraphics[width=\textwidth]{figs/5/sen1_mesh.png}
        \end{subfigure}
        \hspace{1.5em}
        \begin{subfigure}[b]{0.19\textwidth}
          \centering
          \includegraphics[width=\textwidth]{figs/5/sen2_mesh.png}
        \end{subfigure}
        \hspace{1.5em}
        \begin{subfigure}[b]{0.19\textwidth}
          \centering
          \includegraphics[width=\textwidth]{figs/5/Lshaped_mesh.png}
        \end{subfigure}
        \hspace{1.5em}
        \begin{subfigure}[b]{0.19\textwidth}
          \centering
          \includegraphics[width=\textwidth]{figs/5/holed_mesh.png}
        \end{subfigure}\\
        \begin{subfigure}[b]{0.19\textwidth}
          \centering
          \includegraphics[width=\textwidth]{figs/5/sen1_bc.png}
        \end{subfigure}
        \hspace{1.5em}
        \begin{subfigure}[b]{0.19\textwidth}
          \centering
          \includegraphics[width=\textwidth]{figs/5/sen2_bc.png}
        \end{subfigure}
        \hspace{1.5em}
        \begin{subfigure}[b]{0.19\textwidth}
          \centering
          \includegraphics[width=\textwidth]{figs/5/Lshaped_bc.png}
        \end{subfigure}
        \hspace{1.5em}
        \begin{subfigure}[b]{0.19\textwidth}
          \centering
          \includegraphics[width=\textwidth]{figs/5/holed_bc.png}
        \end{subfigure}
        \\\caption*{Authors.}
      \end{figure}

\end{frame}

\subsection{Single Edge Notch Mode I}
\begin{frame}{}
    \frametitle{Single Edge Notch Mode I}
    \vspace{-1em}
    \begin{columns}
        \begin{column}{0.5\textwidth}
            \begin{figure}
                \centering
                \caption{Geometry and boundary conditions.}
                \begin{subfigure}[b]{0.45\textwidth}
                    \centering
                    \includegraphics[width=\textwidth]{figs/5/sen1_mesh.png}
                    \end{subfigure}
                \begin{subfigure}[b]{0.45\textwidth}
                    \centering
                    \includegraphics[width=\textwidth]{figs/5/sen1_bc.png}
                \end{subfigure}
                \\\fonte{ \citeonline{Ferreira2024}.}
            \end{figure}
        \end{column}
        \begin{column}{0.55\textwidth}
        Imposed cyclic displacement
            \[
                \Delta \bar{u}_n =
                \begin{cases}
                4 \times 10^{-3} \text{ mm} & \text{if } 1 \leq n \leq 20 \\
                -4 \times 10^{-4} \text{ mm} & \text{if } 21 \leq n \leq 60 \\
                4 \times 10^{-4} \text{ mm} & \text{if } 61 \leq n \leq 80
                \end{cases} 
                .
            \]
        \end{column}
    \end{columns}
    \begin{table}
        \centering
        \caption{Physical properties for benchmark examples.}
        \begin{tabular}{c@{\hspace{3.5em}}c@{\hspace{3.5em}}c@{\hspace{3.5em}}c@{\hspace{3.5em}}c}
        \hline
        Example & $E$ GPa & $\nu$ & $G_c$ N/mm & $\ell_0$ mm   \\
        \hline
        SEN Mode I & 210 & 0.3 & 2.7 & 0.01 \\
        \hline
        \end{tabular}
        \fonte{Authors.}
        \label{tab:numerical_tests}
    \end{table} 
\end{frame}


\begin{frame}{}
  \vspace{-2em}
    \begin{columns}[t] % Add top alignment
        \begin{column}{0.55\textwidth} % Reduce width
            \begin{figure}
            \centering
            \caption{Specimen with load reversal.}
            
            \begin{subfigure}[b]{0.4\textwidth} 
                \centering
                \includegraphics[width=\textwidth]{figs/5/damageScale.png}
            \end{subfigure}

            \begin{subfigure}[b]{0.55\textwidth} 
                \centering
                \animategraphics[loop,autoplay,width=\textwidth]{10}{figs/5/sen1/result.}{0000}{0079}            
            \end{subfigure}
            \\\fonte{Authors.}
            \end{figure}
        \end{column}
        \hspace{-3em} 
        \begin{column}{0.5\textwidth} 
          \vspace{2em} 
          \begin{figure}
            \caption{Reaction forces.}
            \centering
            \animategraphics[loop,autoplay,width=0.9\textwidth]{10}{figs/5/sen1/frame_}{0002}{0081}            
            \\\fonte{Authors.}
          \end{figure}
        \end{column}
      \end{columns}
\end{frame}


\subsection{Single Edge Notch Mode II}
\begin{frame}{}
    \frametitle{Single Edge Notch Mode II}
    \vspace{-1em}
    \begin{columns}
        \begin{column}{0.5\textwidth}
            \begin{figure}
                \centering
                \caption{Geometry and boundary conditions.}
                \begin{subfigure}[b]{0.45\textwidth}
                    \centering
                    \includegraphics[width=\textwidth]{figs/5/sen2_mesh.png}
                \end{subfigure}
                \begin{subfigure}[b]{0.45\textwidth}
                    \centering
                    \includegraphics[width=\textwidth]{figs/5/sen2_bc.png}
                \end{subfigure}
                \\\fonte{ \citeonline{Ferreira2024}.}
            \end{figure}
        \end{column}
        \begin{column}{0.55\textwidth}
            The imposed cyclic loading for this case is:
            \[
                \Delta \bar{u}_n =
                \begin{cases}
                1 \times 10^{-3} \text{ mm} & \text{if } 1 \leq n \leq 6 \\
                3 \times 10^{-4} \text{ mm} & \text{if } 7 \leq n \leq 26  \\
                -3 \times 10^{-4} \text{ mm} & \text{if } 27 \leq n \leq 106  \\
                3 \times 10^{-4} \text{ mm} & \text{if } 107 \leq n \leq 146 
                \end{cases} 
                 \quad .
            \]
        \end{column}
    \end{columns}
    \begin{table}
        \centering
        \caption{Physical properties for benchmark examples.}
        \begin{tabular}{c@{\hspace{3.5em}}c@{\hspace{3.5em}}c@{\hspace{3.5em}}c@{\hspace{3.5em}}c}
        \hline
        Example & $E$ GPa & $\nu$ & $G_c$ N/mm & $\ell_0$ mm   \\
        \hline
        SEN Mode II & 210 & 0.3 & 2.7 & 0.01 \\
        \hline
        \end{tabular}
        \fonte{Authors.}
        \label{tab:numerical_tests}
    \end{table} 
\end{frame}

\begin{frame}{}
  \vspace{-2em}
    \begin{columns}[t] % Add top alignment
        \begin{column}{0.55\textwidth} % Reduce width
            \begin{figure}
            \centering
            \caption{Specimen with load reversal.}
            
            \begin{subfigure}[b]{0.4\textwidth} 
                \centering
                \includegraphics[width=\textwidth]{figs/5/damageScale.png}
            \end{subfigure}

            \begin{subfigure}[b]{0.55\textwidth} 
                \centering
                \animategraphics[loop,autoplay,width=\textwidth]{10}{figs/5/sen2/result.}{0000}{0145}            
            \end{subfigure}
            \\\fonte{Authors.}
            \end{figure}
        \end{column}
        \hspace{-3em} 
        \begin{column}{0.5\textwidth} 
          \vspace{2em} 
          \begin{figure}
            \caption{Reaction forces.}
            \centering
            \animategraphics[loop,autoplay,width=0.9\textwidth]{10}{figs/5/sen2/frame_}{0002}{0147}            
            \\\fonte{Authors.}
          \end{figure}
        \end{column}
      \end{columns}
\end{frame}


\subsection{L-Shaped Specimen}
\begin{frame}{}
    \frametitle{L-Shaped Specimen}
    \vspace{-1em}
    \begin{columns}
        \begin{column}{0.5\textwidth}
            \begin{figure}
                \centering
                \caption{Geometry and boundary conditions.}
                \begin{subfigure}[b]{0.42\textwidth}
                    \centering
                    \includegraphics[width=\textwidth]{figs/5/Lshaped_mesh.png}
                \end{subfigure}
                \begin{subfigure}[b]{0.42\textwidth}
                    \centering
                    \includegraphics[width=\textwidth]{figs/5/Lshaped_bc.png}
                \end{subfigure}
                \caption{ \citeonline{Ferreira2024}.}
            \end{figure}
        \end{column}
        \begin{column}{0.55\textwidth}
            Subjected to the following displacement history:
            \[
                \Delta \bar{u}_n =
                \begin{cases}
                1 \times 10^{-2} \text{ mm} & \text{if } 1 \leq n \leq 36  \\
                -3 \times 10^{-2} \text{ mm} & \text{if } 37 \leq n \leq 48  \\
                -1 \times 10^{-2} \text{ mm} & \text{if } 49 \leq n \leq 84  \\
                3 \times 10^{-2} \text{ mm} & \text{if } 85 \leq n \leq 96  
                \end{cases} 
                \quad .
            \]
        \end{column}
    \end{columns}   
    \vspace{-1em}
    \begin{table}
        \centering
        \caption{Physical properties for benchmark examples.}
        \begin{tabular}{c@{\hspace{3.5em}}c@{\hspace{3.5em}}c@{\hspace{3.5em}}c@{\hspace{3.5em}}c}
        \hline
        Example & $E$ GPa & $\nu$ & $G_c$ N/mm & $\ell_0$ mm   \\
        \hline
        L-Shaped & 25.85 & 0.18 & 0.095 & 5.0 \\
        \hline
        \end{tabular}
        \fonte{Authors.}
        \label{tab:numerical_tests}
    \end{table} 
\end{frame}

\begin{frame}{}
  \vspace{-2em}
    \begin{columns}[t] % Add top alignment
        \begin{column}{0.55\textwidth} % Reduce width
            \begin{figure}
            \centering
            \caption{Specimen with load reversal.}
            
            \begin{subfigure}[b]{0.4\textwidth} 
                \centering
                \includegraphics[width=\textwidth]{figs/5/damageScale.png}
            \end{subfigure}

            \begin{subfigure}[b]{0.55\textwidth} 
                \centering
                \animategraphics[loop,autoplay,width=\textwidth]{10}{figs/5/Lshaped/result.}{0000}{0095}            
            \end{subfigure}
            \\\fonte{Authors.}
            \end{figure}
        \end{column}
        \hspace{-3em} 
        \begin{column}{0.5\textwidth} 
          \vspace{2em} 
          \begin{figure}
            \caption{Reaction forces.}
            \centering
            \animategraphics[loop,autoplay,width=0.9\textwidth]{10}{figs/5/Lshaped/frame_}{0002}{0097}            
            \\\fonte{Authors.}
          \end{figure}
        \end{column}
      \end{columns}
\end{frame}


\subsection{Holed Plate Specimen}
\begin{frame}{}
    \frametitle{Holed Plate Specimen}
    \vspace{-1em}
    \begin{columns}
        \begin{column}{0.5\textwidth}
            \begin{figure}
                \centering
                \caption{Geometry and boundary conditions.}
                \begin{subfigure}[b]{0.45\textwidth}
                    \centering
                    \includegraphics[width=\textwidth]{figs/5/holed_mesh.png}
                \end{subfigure}
                \begin{subfigure}[b]{0.45\textwidth}
                    \centering
                    \includegraphics[width=\textwidth]{figs/5/holed_bc.png}
                \end{subfigure}
                \\\caption{ \citeonline{Ferreira2024}.}
            \end{figure}
        \end{column}    
        \begin{column}{0.55\textwidth}
            The imposed cyclic loading is specified as:
            \[
                \Delta \bar{u}_n =
                \begin{cases}
                1 \times 10^{-2} \text{ mm} & \text{if } 1 \leq n \leq 301 \\
                3 \times 10^{-2} \text{ mm} & \text{if } 302 \leq n \leq 903 
                \end{cases} 
                \quad .
            \]
        \end{column}
    \end{columns}
    \vspace{-1em}
    \begin{table}
        \centering
        \caption{Physical properties for benchmark examples.}
        \begin{tabular}{c@{\hspace{3.5em}}c@{\hspace{3.5em}}c@{\hspace{3.5em}}c@{\hspace{3.5em}}c}
        \hline
        Example & $E$ GPa & $\nu$ & $G_c$ N/mm & $\ell_0$ mm   \\
        \hline
        Holed Plate & 210 & 0.3 & 2.7 & 0.02 \\
        \hline
        \end{tabular}
        \fonte{Authors.}
        \label{tab:numerical_tests}
    \end{table} 
\end{frame}


\begin{frame}{}
  \vspace{-2em}
    \begin{columns}[t] % Add top alignment
        \begin{column}{0.55\textwidth} % Reduce width
            \begin{figure}
            \centering
            \caption{Specimen with load reversal.}
            
            \begin{subfigure}[b]{0.4\textwidth} 
                \centering
                \includegraphics[width=\textwidth]{figs/5/damageScale.png}
            \end{subfigure}

            \begin{subfigure}[b]{0.55\textwidth} 
                \centering
                \animategraphics[loop,autoplay,width=\textwidth]{10}{figs/5/holed/result.}{0003}{0102}            
            \end{subfigure}
            \\\fonte{Authors.}
            \end{figure}
        \end{column}
        \hspace{-3em} 
        \begin{column}{0.5\textwidth} 
          \vspace{2em} 
          \begin{figure}
            \caption{Reaction forces.}
            \centering
            \animategraphics[loop,autoplay,width=0.9\textwidth]{10}{figs/5/holed/frame_}{0002}{0101}            
            \\\fonte{Authors.}
          \end{figure}
        \end{column}
      \end{columns}
\end{frame}



