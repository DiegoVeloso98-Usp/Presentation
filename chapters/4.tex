\section{Computational Implementation}

\begin{frame}
  \centering
  \vfill
  \begin{beamercolorbox}[sep=14pt,center,rounded=true,shadow=true,wd=\textwidth]{frametitle}
    \usebeamerfont{frametitle}\LARGE Computational Implementation
  \end{beamercolorbox}
  \vfill
\end{frame}


\subsection{High-Performance Computing Tools}

\begin{frame}{}
    \frametitle{High-Performance Computing Tools}
    The following high-performance computing tools are utilized in the implementation:
    \begin{itemize}[label=\textbullet]
        \item \textbf{PETSc}: A suite of data structures and routines for the scalable (parallel) solution of scientific applications modeled by partial differential equations.
        \item \textbf{METIS}: A software package for partitioning unstructured graphs and computing fill-reducing orderings of sparse matrices.
        \item \textbf{GMSH}: A 3D finite element mesh generator with built-in pre- and post-processing facilities.
        \item \textbf{HDF5}: A data model, library, and file format for storing, managing and built for fast I/O processing and storage.
        \item \textbf{Paraview}: An open-source, multi-platform data analysis and visualization application.
    \end{itemize}
\end{frame}


% \subsection{PETSc}
\begin{frame}{}
    \frametitle{PETSc}
    PETSc (Portable, Extensible Toolkit for Scientific Computation) is designed for the \textbf{scalable} 
    solution of scientific applications modeled by partial differential equations (\textbf{PDEs}). 
    \begin{figure}[H]
      \centering
        \includegraphics[width=.5\textwidth]{figs/4/petsc.png}
    \end{figure}
    

    \textbf{It is a powerful and convenient interface for}:

    \begin{itemize}[label=\textbullet]
        \item Management of \textbf{parallelism and communication}.
        \item \textbf{Algebraic operations} on distributed vectors (\texttt{Vec}) and matrices(\texttt{Mat}).
        \item Linear (\texttt{KSP}) and nonlinear (\texttt{SNES}) \textbf{solvers} (algorithms (NR, LS, TR, etc) based on \texttt{KSP} objects).
    \end{itemize}


\end{frame}

\begin{frame}{}
    \frametitle{METIS}
    METIS is a library for partitioning and producing fill reducing orderings for finite element meshes.
    \begin{columns}[t]
      \begin{column}{0.45\textwidth}
        \begin{itemize}[label=\textbullet]
          \item \textbf{Fill-reducing}: \texttt{METIS\_NodeND}.
        \end{itemize}
        \begin{figure}[H]
        \centering
        \caption{Sparse matrix representation.}
            \includegraphics[width=1.1\textwidth]{figs/4/sen2_matrix_REFxRCMxMETIS.png}
        \\\caption*{Authors.}
        \label{fig:matrix-sen2}
        \end{figure}
    \end{column}
    % \hspace{2em}
    \begin{column}{0.55\textwidth}
        \begin{itemize}[label=\textbullet]
          \item \textbf{Mesh Partitioning}: \texttt{METIS\_PartMeshDual}.
        \end{itemize}
        \begin{figure}[H]
        \centering
        \caption{Mesh partitioning.}
          \begin{subfigure}[b]{0.4\textwidth}
          \centering
          \includegraphics[width=\textwidth]{figs/4/processScale.png}
        \end{subfigure}
        \\
          \begin{subfigure}[b]{0.15\textwidth}
          \centering
          \includegraphics[width=\textwidth]{figs/4/sen1_mesh.png}
        \end{subfigure}
        \begin{subfigure}[b]{0.15\textwidth}
          \centering
          \includegraphics[width=\textwidth]{figs/4/sen2_mesh.png}
        \end{subfigure}
        \begin{subfigure}[b]{0.15\textwidth}
          \centering
          \includegraphics[width=\textwidth]{figs/4/Lshaped_mesh.png}
        \end{subfigure}
        \begin{subfigure}[b]{0.15\textwidth}
          \centering
          \includegraphics[width=\textwidth]{figs/4/holed_mesh.png}
        \end{subfigure}
        \\
        \begin{subfigure}[b]{0.19\textwidth}
          \centering
          \includegraphics[width=\textwidth]{figs/4/sen1_MetisDualGraph.png}
        \end{subfigure}
        \begin{subfigure}[b]{0.19\textwidth}
          \centering
          \includegraphics[width=\textwidth]{figs/4/sen2_MetisDualGraph.png}
        \end{subfigure}
        \begin{subfigure}[b]{0.15\textwidth}
          \centering
          \includegraphics[width=\textwidth]{figs/4/Lshaped_MetisDualGraph.png}
        \end{subfigure}
        \begin{subfigure}[b]{0.15\textwidth}
          \centering
          \includegraphics[width=\textwidth]{figs/4/holed_MetisDualGraph.png}
        \end{subfigure}
        \\\caption*{Authors.}
        \label{fig:process-partitioning}
      \end{figure}

    \end{column}
  \end{columns}
\end{frame}

% \subsection{Additional Tools}
\begin{frame}{}
    \frametitle{C++, HDF5, Paraview, GMSH, Python}

    \begin{itemize}[label=\textbullet]
      \item \textbf{C++}: The programming language used for the implementation.
      \item \textbf{HDF5}: Built for fast I/O processing and storage.
      \item \textbf{Paraview}: Open-source multiple-platform application for interactive, scientific visualization.
      \item \textbf{GMSH}: A 3D finite element mesh generator with built-in pre- and post-processing facilities.
      \item \textbf{Python}: A programming language used for pre- and post-processing and for the GMSH API.
    \end{itemize}

    \begin{figure}[H]
      \centering
      \begin{subfigure}[b]{0.08\textwidth}
        \centering
        \includegraphics[width=\textwidth]{figs/4/cpp.png}
      \end{subfigure}
      \hspace{1em}
      \begin{subfigure}[b]{0.11\textwidth}
        \centering
        \includegraphics[width=\textwidth]{figs/4/hdf5.png}
      \end{subfigure}
      \hspace{1em}
      \begin{subfigure}[b]{0.37\textwidth}
        \centering
        \includegraphics[width=\textwidth]{figs/4/paraview.png}
      \end{subfigure}
      \hspace{1em}
      \begin{subfigure}[b]{0.09\textwidth}
        \centering
        \includegraphics[width=\textwidth]{figs/4/gmsh_title.png}
      \end{subfigure}
      \hspace{1em}
      \begin{subfigure}[b]{0.07\textwidth}
        \centering
        \includegraphics[width=\textwidth]{figs/4/py.png}
      \end{subfigure}
  \end{figure}

\end{frame}




\subsection{Global Minimization}
\begin{frame}{}
    \frametitle{Solution of the coupled problem}
  \begin{itemize}[label=\textbullet]
    \item \textbf{Monolithic approach} \cite{bharali2022}
  \end{itemize}
  \begin{equation}
      \begin{bmatrix}
      \bm{K}^{uu} & \bm{K}^{ud} \\
      \bm{K}^{du} & \bm{K}^{dd}
      \end{bmatrix}
      \begin{bmatrix}
      \bm{\Delta u} \\
      \Delta d
      \end{bmatrix}
      =
      \begin{bmatrix}
      \bm{r^u} \\
      \bm{r^d}
      \end{bmatrix}
  \end{equation}
  \begin{itemize}[label=\textbullet]
    \item \textbf{Staggered approach}
  \end{itemize}
  \begin{equation}
      \begin{bmatrix}
      \bm{K}^{uu} & 0 \\
      0 & \bm{K}^{dd}
      \end{bmatrix}
      \begin{bmatrix}
      \bm{\Delta u} \\
      \Delta d
      \end{bmatrix}
      =
      \begin{bmatrix}
      \bm{r^u} \\
      \bm{r^d}
      \end{bmatrix}
  \end{equation}
  or, in more detail,
  \begin{equation}
    \begin{cases}
    \partial_u \varPi_{n+1}(\bm{u}^{i+1}_{n+1},d^{i}_{n+1}) = 0 \\
    \partial_d \varPi_{n+1}(\bm{u}^{i+1}_{n+1},d^{i+1}_{n+1}) \left[\Delta d^{i+1}_{n+1}\right] = 0
    , \quad
    \partial_d \varPi_{n+1}(\bm{u}^{i+1}_{n+1},d^{i+1}_{n+1}) \geq 0
    , \quad
    \Delta d^{i+1}_{n+1} \geq 0
    \end{cases}
    .
    \label{eq:alternated_minimization}
\end{equation} 


\end{frame}



\begin{frame}{}
    \frametitle{Staggered approach} 
\begin{algorithmic}  [1] %adds line numbers
\Require Load solution $(\mathbf{u}_n, d_n)$ from step $n$ and boundary conditions $g_{n+1}, t_{n}$ at current step $n$
\State initialize $i \gets 0$
\State set $(\mathbf{u}^{0}, d^{0}) \gets (\mathbf{u}_{n}, d_{n})$
\While{$Re_{\text{stag}} \geq TOL_{\text{stag}}$}
  \State $i \gets i + 1$
  \State given $\,d^{\,i-1} = d_{n} + \Delta d^{\,i-1}\,$, find \,\(\mathbf{u}^{\,i}\) solving \,
  \(
    \partial_{\mathbf{u}}\varPi_{\,n+1}(\mathbf{u}^{\,i}, d^{\,i-1}) = 0
  \)
  \State given \,\(\mathbf{u}^{\,i}\), find \,\(\Delta d^{\,i}\) solving \,
  \(\partial_{d}\varPi_{\,n+1}(u^{\,i}, d_{\,n})[\Delta d^{\,i}]=0\)
  with
  \(\qquad \qquad \partial_{d}\varPi_{\,n+1}(\mathbf{u}^{\,i}, d_{\,n}) \ge 0, \quad \Delta d^{\,i} \ge 0\)
  \State compute \(\left|Re_{\text{stag}} = \partial_{\mathbf{u}}\varPi_{\,n+1}(\mathbf{u}^{\,i}, d^{\,i})[\Delta \mathbf{u}^{\,i}]\right|\)
\EndWhile
\State $(\mathbf{u}_{n+1}, d_{n+1}) \gets (\,\mathbf{u}^{\,i}, d^{\,i}\,)$
\end{algorithmic}

\end{frame}

\subsection{Linear Momentum Solution}

\begin{frame}
\frametitle{Linear Momentum Solution}
% \vspace{-2em}
From Principle of Virtual Work, the weak form is given as:
  \begin{equation}
      \int_{\Omega} \left[\bm{\nabla^s w}:\mathbb{D} \bm{\nabla^s u}\right]dV 
      = \int_{\Gamma_t} \bm{t}\cdot\bm{w}dS 
      + \int_{\mathcal{B}} \bm{b}\cdot\bm{w}dV
      \qquad \forall \text{ \(\bm{w} = 0\) on \(\Gamma_u\)}
  \end{equation}
  \vspace{-3em}
   \begin{columns}[t]
      \begin{column}{0.55\textwidth}
        \begin{equation}
          \begin{cases}
            \bm{u}(x,y) = \bm{N^e}(x,y)\bm{u}^e \\
            \bm{w}(x,y) = \bm{N^e}(x,y)\bm{w}^e 
          \end{cases}
            \forall (x,y) \in \Omega_e
        \end{equation}
        \begin{equation}
          \bm{\varepsilon} = \bm{\nabla^s u} 
          = \bm{B^e}\bm{u^e}
        \end{equation}
        and the system of equations can be written as
        \begin{equation}
            \bm{K}\bm{u} = \bm{F}
        \end{equation}
    \end{column}
    % \hspace{2em}
    \begin{column}{0.46\textwidth}
      \begin{figure}[H]
        \centering
        \caption{Discrete domain.}
        \includegraphics[width=0.4\textwidth]{figs/4/finite_body.png}
        \\\caption*{\citeonline{anand_continuum_2020}.}
      \end{figure}
    \end{column}
  \end{columns}
  \vspace{-1em}
  where
  \vspace{-1em}
  \begin{equation}
    \bm{K} = \sum_{e=1}^n  \left[\int_{\Omega_e} \bm{B}^T_e\mathbb{C}\bm{B}_e d\Omega\right] \bm{u}^e
    \quad \text{and} \quad
    \bm{F} = \sum_{e=1}^n \int_{\varGamma_e} \bm{N}^T_e\bm{t}d\varGamma
            + \sum_{e=1}^n \int_{\Omega_e} \bm{N}^T_e\bm{b}d\Omega
  \end{equation}

\end{frame}

\begin{frame}{}
    \frametitle{Newton Raphson}
    Taylor expansion gives
\begin{equation}
    \bm{R}(\bm{u}^{k+1}) 
    = \bm{R}(\bm{u}^k) 
    + \left[\frac{\partial \bm{R}}{\partial \bm{u}}\right]_{\bm{u}^k}\left(\bm{u}^{k+1} - \bm{u}^k\right)
    + \mathcal{O}\left(\left(\bm{u}^{k+1} - \bm{u}^k\right)^2\right)
\end{equation}
that, after linearization and assuming that \(\bm{R}(\bm{u}^{k+1}) = 0\)
\begin{equation}
    \bm{R}(\bm{u}^k) 
    + \left[\frac{\partial \bm{R}}{\partial \bm{u}}\right]_{\bm{u}^k}\left(\bm{u}^{k+1} - \bm{u}^k\right)
    = 0
    \quad ,
\end{equation}
Rearranging the equation, we can write
\begin{equation}
    \left[\frac{\partial^2 \varPi}{\partial \bm{u}^2}\right]_{\bm{u}^k}\bm{\Delta u}^{k+1} 
    = -\bm{R}(\bm{u}^k)
    \quad \Rightarrow \quad
        \bm{K(\bm{u}^k)}\bm{\Delta u}^{k+1} = -\bm{R}(\bm{u}^k)
\end{equation}
\texttt{PCFactorSetReuseOrdering(PETSC\_TRUE)} and \texttt{KSPSetReusePreconditioner(PETSC\_FALSE)}

\end{frame}

\begin{frame}{}
    \frametitle{Line Search}
    Optimal step size
\begin{equation}
    \bm{u}^{k+1}_{n} = \bm{u}^k + \eta_{n} \bm{\Delta u}^{k+1} \quad .
    \label{eq:line_search}
\end{equation}
assuming that \(\bm{R}(\eta_{n}) \perp \bm{\Delta u}\)
\begin{equation}
    R(\eta_{n}) = \bm{R}(\eta_{n})\cdot\bm{\Delta u} = \bm{R}(\bm{u}^k + \eta_{n}\bm{\Delta u})\cdot\bm{\Delta u}
\end{equation}
that, after some algebra, leads to the following iterative formula for \(\eta\):
\begin{equation}
    \eta_{n+1} = 
    \begin{cases}
        \dfrac{\alpha_{n}}{2} + \sqrt{\left(\dfrac{\alpha_{n}}{2}\right)^2 - \alpha_n } & \text{if } \alpha_n < 0 \\[0.5ex]
        \dfrac{\alpha_{n}}{2} & \text{if } \alpha_n > 0
    \end{cases}
    \quad \rightarrow \quad \text{where} \quad
        \alpha_n = \dfrac{R(\bm{u}^k)}{ R(\eta_n) } = \dfrac{R_0}{R(\eta_n)}
\end{equation}
with a stopping criterion defined by \citeonline{bonet_nonlinear_2016} as 
\begin{equation}
    \left|R(\eta_{n})\right| < \rho \left| R(0)\right| \quad ,
    \qquad \text{which} \quad
    \rho = 0.5
\end{equation}

\end{frame}



\subsection{Damage Solution}
\begin{frame}{}
    \frametitle{Symmetric Linear Complementarity Problem (SLCP)}
    \vspace{-2em}
    \citeonline{Marengo2021} performs a Taylor expansion around \(\hat{\bm{d}}_n\)
    \begin{equation}
    \varPi_{n+1}(\hat{\bm{u}}^i,\hat{\bm{d}})
    = \underbrace{\frac12\,\Delta\hat{\bm{d}}^{\mathsf T}\,Q^i\,\Delta\hat{\bm{d}}
    + \Delta\hat{\bm{d}}^{\mathsf T}\,q^i}_{\text{Term to be minimized}}
    + \varPi_{n+1}(\hat{\bm{u}}^i,\hat{\bm{d}}_n),
\end{equation}
where
\begin{equation}
    \Delta\hat{\bm{d}} = \hat{\bm{d}} - \hat{\bm{d}}_n
    \, ,\quad
    Q^i = \nabla^2_{\hat{\bm{d}}\hat{\bm{d}}} \varPi_{n+1}(\hat{\bm{u}}^i,\hat{\bm{d}}_n)
    \, ,\quad
    q^i = \nabla_{\hat{\bm{d}}}\,\varPi_{n+1}(\hat{\bm{u}}^i,\hat{\bm{d}}_n)\,.
    \label{eq:Qq}
\end{equation}
and finally, the Jacobian (Hessian) \(Q^i\) and Gradient(Residue) \(q^i\) are given by
\begin{equation}
  Q^i := \Psi_e(\hat{\bm{u}}^i) \;+\; G_c\,\Phi_e,
  \qquad
  q^i := Q^i\,\hat{\bm{d}}_n \;-\; \psi_e(\hat{\bm{u}}^i).
\end{equation}


\end{frame}


\begin{frame}{}
    \frametitle{Symmetric Linear Complementarity Problem (SLCP)}
    Where, the constant element dissipation matrix \(\Phi_e\) is given by
\begin{equation}
  \Phi_e := 
  \begin{cases}
    \displaystyle\int_{\Omega_e}
    \left(\frac{3l_0}{8}\,B_{d,e}^{\mathsf T}B_{d,e}\right)
    \,\mathrm{d}\Omega_e & \text{if AT1} \\[2ex]
    \displaystyle\int_{\Omega_e}
    \left(l_0^{-1}\,N_{d,e}^{\mathsf T}N_{d,e}
            + l_0\,B_{d,e}^{\mathsf T}B_{d,e}\right)
    \,\mathrm{d}\Omega_e & \text{if AT2}
  \end{cases}
\end{equation}
and the element free energy matrix \(\Psi_e\) and vector \(\psi_e\) are given by
\begin{equation}
\Psi_e(\hat{\bm{u}}^i_e)
    := \int_{\Omega_e}
    2\,\psi_0^+\bigl(\hat{\bm{u}}^i_e\bigr)\,
    N_{d,e}^{\mathsf T}N_{d,e}
    \,\mathrm d\Omega_e
\end{equation}

\begin{equation}
\psi_e(\hat{\bm{u}}^i_e)
    := 
    \begin{cases}
        \displaystyle \int_{\Omega_e}
        \left[2\,\psi_0^+\bigl(\hat{\bm{u}}^i_e\bigr) - \dfrac{3l_0}{8}\right]\,
        N_{d,e}^{\mathsf T}
        \,\mathrm d\Omega_e. & \text{if AT1} \\[2ex]
        \displaystyle \int_{\Omega_e}
        2\,\psi_0^+\bigl(\hat{\bm{u}}^i_e\bigr)\,
        N_{d,e}^{\mathsf T}
        \,\mathrm d\Omega_e. & \text{if AT2}
    \end{cases}
\end{equation}


\end{frame}

\begin{frame}{}
    \frametitle{Projected Successive Over-Relaxation (PSOR)}
The SLCP is gives rise to the following conditions:
    \begin{equation}
    \partial_d \varPi_{n+1}(\bm{u}_{n+1},d_{n+1}) \left[\Delta d_{n+1}\right]
    , \quad
    \partial_d \varPi_{n+1}(\bm{u}_{n+1},d_{n+1}) \geq 0
    , \quad
    \Delta d_{n+1} \geq 0
    \end{equation}
    which can be solved by a Projected Successive Over-Relaxation (PSOR) iterative scheme
    \begin{equation}
    x^{k+1} = \left(\mathbf{L} + \mathbf{D}\right)^{-1}\left(b - \mathbf{U}x^k\right)
\end{equation}
which, after some algebra, \citeonline{Marengo2021} shows that can be written as
\begin{equation}
    \Delta d^{k}_{r} =
    \left\langle \Delta d^{k-1}_{r} - D^{-1}_{rr}\left[Q_{rc}\Delta d^{k-1}_{c}
    + L_{rc}\left(\Delta d^{k}_{c} - \Delta d^{k-1}_{c}\right)\right] \right\rangle_+
\end{equation}
where
\begin{columns}
  \begin{column}{0.4\textwidth}
    \begin{equation}
      \begin{cases}
        L_{rc} := Q_{r>c} \\[2ex]
        D_{rr} := Q_{r=r}
      \end{cases}
    \end{equation}
  \end{column}
  \begin{column}{0.4\textwidth}
    \begin{equation}
        \langle x \rangle_+ := 
      \begin{cases}
        x & \text{if } x \geq 0 \\[2ex]
        0 & \text{if } x < 0
      \end{cases}
    \end{equation}
  \end{column}
\end{columns} 



\end{frame}


